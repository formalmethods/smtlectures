\subsection{Introduction}

\begin{frame}
  \frametitle{Simplex Algorithm}

  Invented by Tobias Dantzig around 1950 \\
  \vfill
  Used to solve optimization problems in linear programming \\
  \vfill
  Greg Nelson was the first to employ it for 
  constraint solving, AFAIK, around 1980
  \vfill
  Difference is that in linear programming input problem is
  feasible and one looks for optimum. In constraint solving
  problem can be infeasible, and we are interested in 
  finding any solution

\end{frame}

\begin{frame}
  \frametitle{Introduction}

Linear Rational Arithmetic \Lra consists in solving
Boolean combinations of atoms of the form
$$
\sum_{j=1}^n a_j x_j \leq b\quad\quad\quad\quad\sum_{j=1}^n a_j x_j \geq b
$$
where $a_j$ are constants (coefficients), $x_j$ are
variables, and $b$ is a constant (bound). The domain
of $a_j$, $x_j$, $b$ is that of rationals.
\vfill
\pause
Notice that the following translations hold
\begin{itemize}
  \item $\sum_{j=1}^n a_j x_j = b     \quad\quad\Longrightarrow\quad\quad  (\sum_{j=1}^n a_j x_j \leq  b) \wedge (\sum_{j=1}^n a_j x_j \geq b)$  
  \item $\sum_{j=1}^n a_j x_j < b     \quad\quad\Longrightarrow\quad\quad$ see later
  \item $\sum_{j=1}^n a_j x_j > b     \quad\quad\Longrightarrow\quad\quad$ see later 
  \item $\sum_{j=1}^n a_j x_j \not= b \quad\quad\Longrightarrow\quad\quad  (\sum_{j=1}^n a_j x_j < b) \vee (\sum_{j=1}^n a_j x_j > b)$  
\end{itemize}

\end{frame}

\subsection{Preprocessing}

\begin{frame}
  \frametitle{Preprocessing}

  In the SMT setting, we are given a formula $\varphi$ like
  $$
  (x \geq 0) \wedge ((x + y \leq 2) \vee (x + 2y - z \geq 6)) \wedge ((x + y \geq 2) \vee (2y - z \leq 4))
  $$
  We perform a {\bf preprocessing step}, in order to separate the formula into a set of {\bf equations}
  and a set of simple {\bf bounds}. This is done by introducing {\bf fresh} variables. 
  \pause
  \vfill
  The formula above $\varphi$ is equivalent to (the conjunction of)
  $$
  \begin{array}{lc}
    (x \geq 0) \wedge ((s_1 \leq 2) \vee (s_2 \geq 6)) \wedge ((s_1 \geq 2) \vee (s_3 \leq 4)) & \onslide<3>{\varphi'} \\
    \\
    (s_1 = x + y) \\ 
    (s_2 = x + 2y - z) & \onslide<3>{A\vec{x}=\vec{0}} \\
    (s_3 = 2y - z)
  \end{array}
  $$

\end{frame}

\begin{frame}
  \frametitle{Preprocessing}
  In general, from a formula $\varphi$, we end up in a rewritten formula of the kind
  $$
  \varphi' \ \wedge\ A\vec{x} = \vec{0} 
  $$
  where $\varphi'$ is a Boolean combination of {\bf bounds}, while $A\vec{x}=\vec{0}$ is a system of {\bf equations}
  of the form
  $$
  \begin{array}{rcl}
    a_{11} x_1 + \ldots + a_{1n} x_n & = & 0 \\ 
    a_{21} x_1 + \ldots + a_{2n} x_n & = & 0 \\ 
    \ldots \\                             
    a_{i1} x_1 + \ldots + a_{in} x_n & = & 0 \\ 
    \ldots \\                             
    a_{m1} x_1 + \ldots + a_{mn} x_n & = & 0 \\ 
  \end{array}
  $$
  \vfill
  \pause
  Now we detach $A\vec{x} = \vec{0}$ from the formula, and we store it into the \tsolver
  permanently. The \satsolver will work only on $\varphi'$. Therefore {\bf the constraints
  that are pushed into and popped from the \tsolver are just bounds} 

\end{frame}
