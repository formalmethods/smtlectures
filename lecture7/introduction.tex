\begin{frame}
  \frametitle{Introduction}

  Recall from the first lecture: 
  \vfill
  In SMT a theory \T is defined over a {\bf signature} $\Sigma$, which
  is a set of function and predicate symbols.
  The equality symbol $=$ is assumed to be included in every signature.
  \vfill
  The signature of \Uf can include infinite functional and predicate
  symbols 
  $\Sigma = \{ f, g, h, \ldots, p, q, \ldots \}$, which can be used
  as usual to build \tterms using variables
  \vfill
  Examples: 
  $$x = f(g(y), z)\quad\quad\quad p( x, g( x ) )\quad\quad\quad g( y ) \not= g( z )$$
  \vfill
  \Uf is the so-called {\bf empty} theory as it does not contain 
  any ``explicit'' axiom 

\end{frame}

\begin{frame}
  \frametitle{Introduction}

  Still, there is equality $=$. So the following logic rules must be
  respected by any satisfiable \Uf-formula\footnote{Notice that for \Idl and \Lra this was implicitly true.}
  \vfill
  \begin{columns}

  \begin{column}{.3\textwidth}
  \begin{prooftree}
    \def\extraVskip{5pt}
    \AxiomC{}
    \RightLabel{\scriptsize (refl.)}
    \UnaryInfC{$s=s$}
  \end{prooftree}
  \end{column}

  \begin{column}{.3\textwidth}
  \begin{prooftree}
    \def\extraVskip{5pt}
    \AxiomC{$s=t$}
    \RightLabel{\scriptsize (symm.)}
    \UnaryInfC{$t=s$}
  \end{prooftree}
  \end{column}

  \begin{column}{.4\textwidth}
  \begin{prooftree}
    \def\extraVskip{5pt}
    \AxiomC{$s=t \wedge t=r$}
    \RightLabel{\scriptsize (tran.)}
    \UnaryInfC{$s=r$}
  \end{prooftree}
  \end{column}

  \end{columns}
  \vfill
  for all \Uf-terms $s,t,r$ 
  \vfill\pause
  Moreover, there is a further condition that has to hold
  \begin{prooftree}
    \def\extraVskip{5pt}
    \AxiomC{$s_1=t_1 \wedge \ldots \wedge s_n=t_n$}
    \RightLabel{\scriptsize (cong.)}
    \UnaryInfC{$f(s_1,\ldots,s_n)=f(t_1,\ldots,t_n)$}
  \end{prooftree}

\end{frame}
