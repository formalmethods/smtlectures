\subsection{Simplifications}

\begin{frame}
  \frametitle{Simplifications}

  \scriptsize

  We use {\bf simplification} rules to fight the two problems in previous slide,
  before bit-blasting everything
  \vfill
  \pause
  \begin{center}
  \scalebox{.4}{\subsection{Simplifications}

\begin{frame}
  \frametitle{Simplifications}

  \scriptsize

  We use {\bf simplification} rules to fight the two problems in previous slide,
  before bit-blasting everything
  \vfill
  \pause
  \begin{center}
  \scalebox{.4}{\subsection{Simplifications}

\begin{frame}
  \frametitle{Simplifications}

  \scriptsize

  We use {\bf simplification} rules to fight the two problems in previous slide,
  before bit-blasting everything
  \vfill
  \pause
  \begin{center}
  \scalebox{.4}{\subsection{Simplifications}

\begin{frame}
  \frametitle{Simplifications}

  \scriptsize

  We use {\bf simplification} rules to fight the two problems in previous slide,
  before bit-blasting everything
  \vfill
  \pause
  \begin{center}
  \scalebox{.4}{\input{simplification.pdf_t}}
  \end{center}
  \vfill
  \pause
  Simplifications exploit properties of \bitvectors to try to reduce
  the complexity of the formula. We see here some examples, but many
  more rules do exist. Also, it is very important the way they are
  combined together 

\end{frame}

\begin{frame}
  \frametitle{Trivial Simplifications}

  The following are trivial consequences of the semantic of \bitvectors and
  \formulae in general
  \vfill
  \pause
  \bitvectors trivial simplifications
  \begin{itemize}
    \item $t = t\ \Rightarrow\ \top$\ \ \ \ for a generic term
    \item $c = d\ \Rightarrow\ \bot$\ \ \ \ for two different constants $c$ and $d$
    \item $t \band 0\ldots0\ \Rightarrow\ 0\ldots0$\ \ \ \ for a generic term
    \item \ldots
  \end{itemize}
  \vfill
  \pause
  \begin{itemize}
    \item $\varphi \wedge \varphi\ \Rightarrow\ \varphi$\ \ \ \ for a generic formula
    \item $\varphi \wedge \top\ \Rightarrow\ \varphi$\ \ \ \ for a generic formula
    \item $\varphi \vee \top\ \Rightarrow\ \top$\ \ \ \ for a generic formula
    \item \ldots
  \end{itemize}

\end{frame}

\begin{frame}
  \frametitle{Ground Term evaluation}
  
  If a term is {\bf ground}, i.e., it contains no variables, then it
  can be always simplified to a single constant
  \vfill
  \pause
  Examples:
  \begin{itemize}
    \item $0000 :: 1000\ \Rightarrow 00001000$
    \item $0010[1:0]\ \Rightarrow 10$
    \item $0100 + 0101\ \Rightarrow 1001$
  \end{itemize}

\end{frame}

\begin{frame}
  \frametitle{Variable Elimination Rule}

  Suppose that the input formula $\varphi$ is of the kind
  \vfill
  $$
  \varphi'\ \wedge\ (\w{x}{n} = \w{t}{n})
  $$
  \vfill
  where $\w{x}{n}$ is a variable, and $\w{t}{n}$ is a term {\bf not containing} $\w{x}{n}$
  \pause
  \vfill
  then we can rewrite $\varphi$ as
  $$
  \varphi'[\w{t}{n}/\w{x}{n}]
  $$
  \vfill
  i.e., we replace every occurrence of $\w{x}{n}$ by $\w{t}{n}$. \pause We save $n$
  Boolean variables in the reduction to SAT

\end{frame}

\begin{frame}
  \frametitle{Concatenation Elimination Rule}

  Suppose that we have the equality
  \vfill
  $$
  \w{t}{n} :: \w{s}{m} = \w{r}{n} :: \w{u}{m}
  $$
  \vfill
  then, because the concatenations {\bf match}, we can rewrite it as
  \pause
  \vfill
  $$
  (\w{t}{n} = \w{r}{n}) \wedge (\w{s}{m} = \w{u}{m}) 
  $$
  \vfill
  this rewriting may give more opportunity for applications of previous rules

\end{frame}

\begin{frame}
  \frametitle{Exercizes}

  \begin{enumerate}

    \item Complete the missing cases in procedures Bit-Blast-Term and Bit-Blast
    \vfill
    \item Bit-Blast the formula $\neg(\w{x}{3} = 000) \wedge (\w{x}{3} \band \w{y}{3}) = (\w{x}{3} + \w{y}{3})$ 
    \vfill
    \item Simplify the formula $(\w{x}{4} :: \w{y}{4}) = (\w{z}{4} :: \w{x}{4}) \wedge \neg( \w{y}{4} = \w{z}{4})$ 

  \end{enumerate}

\end{frame}
}
  \end{center}
  \vfill
  \pause
  Simplifications exploit properties of \bitvectors to try to reduce
  the complexity of the formula. We see here some examples, but many
  more rules do exist. Also, it is very important the way they are
  combined together 

\end{frame}

\begin{frame}
  \frametitle{Trivial Simplifications}

  The following are trivial consequences of the semantic of \bitvectors and
  \formulae in general
  \vfill
  \pause
  \bitvectors trivial simplifications
  \begin{itemize}
    \item $t = t\ \Rightarrow\ \top$\ \ \ \ for a generic term
    \item $c = d\ \Rightarrow\ \bot$\ \ \ \ for two different constants $c$ and $d$
    \item $t \band 0\ldots0\ \Rightarrow\ 0\ldots0$\ \ \ \ for a generic term
    \item \ldots
  \end{itemize}
  \vfill
  \pause
  \begin{itemize}
    \item $\varphi \wedge \varphi\ \Rightarrow\ \varphi$\ \ \ \ for a generic formula
    \item $\varphi \wedge \top\ \Rightarrow\ \varphi$\ \ \ \ for a generic formula
    \item $\varphi \vee \top\ \Rightarrow\ \top$\ \ \ \ for a generic formula
    \item \ldots
  \end{itemize}

\end{frame}

\begin{frame}
  \frametitle{Ground Term evaluation}
  
  If a term is {\bf ground}, i.e., it contains no variables, then it
  can be always simplified to a single constant
  \vfill
  \pause
  Examples:
  \begin{itemize}
    \item $0000 :: 1000\ \Rightarrow 00001000$
    \item $0010[1:0]\ \Rightarrow 10$
    \item $0100 + 0101\ \Rightarrow 1001$
  \end{itemize}

\end{frame}

\begin{frame}
  \frametitle{Variable Elimination Rule}

  Suppose that the input formula $\varphi$ is of the kind
  \vfill
  $$
  \varphi'\ \wedge\ (\w{x}{n} = \w{t}{n})
  $$
  \vfill
  where $\w{x}{n}$ is a variable, and $\w{t}{n}$ is a term {\bf not containing} $\w{x}{n}$
  \pause
  \vfill
  then we can rewrite $\varphi$ as
  $$
  \varphi'[\w{t}{n}/\w{x}{n}]
  $$
  \vfill
  i.e., we replace every occurrence of $\w{x}{n}$ by $\w{t}{n}$. \pause We save $n$
  Boolean variables in the reduction to SAT

\end{frame}

\begin{frame}
  \frametitle{Concatenation Elimination Rule}

  Suppose that we have the equality
  \vfill
  $$
  \w{t}{n} :: \w{s}{m} = \w{r}{n} :: \w{u}{m}
  $$
  \vfill
  then, because the concatenations {\bf match}, we can rewrite it as
  \pause
  \vfill
  $$
  (\w{t}{n} = \w{r}{n}) \wedge (\w{s}{m} = \w{u}{m}) 
  $$
  \vfill
  this rewriting may give more opportunity for applications of previous rules

\end{frame}

\begin{frame}
  \frametitle{Exercizes}

  \begin{enumerate}

    \item Complete the missing cases in procedures Bit-Blast-Term and Bit-Blast
    \vfill
    \item Bit-Blast the formula $\neg(\w{x}{3} = 000) \wedge (\w{x}{3} \band \w{y}{3}) = (\w{x}{3} + \w{y}{3})$ 
    \vfill
    \item Simplify the formula $(\w{x}{4} :: \w{y}{4}) = (\w{z}{4} :: \w{x}{4}) \wedge \neg( \w{y}{4} = \w{z}{4})$ 

  \end{enumerate}

\end{frame}
}
  \end{center}
  \vfill
  \pause
  Simplifications exploit properties of \bitvectors to try to reduce
  the complexity of the formula. We see here some examples, but many
  more rules do exist. Also, it is very important the way they are
  combined together 

\end{frame}

\begin{frame}
  \frametitle{Trivial Simplifications}

  The following are trivial consequences of the semantic of \bitvectors and
  \formulae in general
  \vfill
  \pause
  \bitvectors trivial simplifications
  \begin{itemize}
    \item $t = t\ \Rightarrow\ \top$\ \ \ \ for a generic term
    \item $c = d\ \Rightarrow\ \bot$\ \ \ \ for two different constants $c$ and $d$
    \item $t \band 0\ldots0\ \Rightarrow\ 0\ldots0$\ \ \ \ for a generic term
    \item \ldots
  \end{itemize}
  \vfill
  \pause
  \begin{itemize}
    \item $\varphi \wedge \varphi\ \Rightarrow\ \varphi$\ \ \ \ for a generic formula
    \item $\varphi \wedge \top\ \Rightarrow\ \varphi$\ \ \ \ for a generic formula
    \item $\varphi \vee \top\ \Rightarrow\ \top$\ \ \ \ for a generic formula
    \item \ldots
  \end{itemize}

\end{frame}

\begin{frame}
  \frametitle{Ground Term evaluation}
  
  If a term is {\bf ground}, i.e., it contains no variables, then it
  can be always simplified to a single constant
  \vfill
  \pause
  Examples:
  \begin{itemize}
    \item $0000 :: 1000\ \Rightarrow 00001000$
    \item $0010[1:0]\ \Rightarrow 10$
    \item $0100 + 0101\ \Rightarrow 1001$
  \end{itemize}

\end{frame}

\begin{frame}
  \frametitle{Variable Elimination Rule}

  Suppose that the input formula $\varphi$ is of the kind
  \vfill
  $$
  \varphi'\ \wedge\ (\w{x}{n} = \w{t}{n})
  $$
  \vfill
  where $\w{x}{n}$ is a variable, and $\w{t}{n}$ is a term {\bf not containing} $\w{x}{n}$
  \pause
  \vfill
  then we can rewrite $\varphi$ as
  $$
  \varphi'[\w{t}{n}/\w{x}{n}]
  $$
  \vfill
  i.e., we replace every occurrence of $\w{x}{n}$ by $\w{t}{n}$. \pause We save $n$
  Boolean variables in the reduction to SAT

\end{frame}

\begin{frame}
  \frametitle{Concatenation Elimination Rule}

  Suppose that we have the equality
  \vfill
  $$
  \w{t}{n} :: \w{s}{m} = \w{r}{n} :: \w{u}{m}
  $$
  \vfill
  then, because the concatenations {\bf match}, we can rewrite it as
  \pause
  \vfill
  $$
  (\w{t}{n} = \w{r}{n}) \wedge (\w{s}{m} = \w{u}{m}) 
  $$
  \vfill
  this rewriting may give more opportunity for applications of previous rules

\end{frame}

\begin{frame}
  \frametitle{Exercizes}

  \begin{enumerate}

    \item Complete the missing cases in procedures Bit-Blast-Term and Bit-Blast
    \vfill
    \item Bit-Blast the formula $\neg(\w{x}{3} = 000) \wedge (\w{x}{3} \band \w{y}{3}) = (\w{x}{3} + \w{y}{3})$ 
    \vfill
    \item Simplify the formula $(\w{x}{4} :: \w{y}{4}) = (\w{z}{4} :: \w{x}{4}) \wedge \neg( \w{y}{4} = \w{z}{4})$ 

  \end{enumerate}

\end{frame}
}
  \end{center}
  \vfill
  \pause
  Simplifications exploit properties of \bitvectors to try to reduce
  the complexity of the formula. We see here some examples, but many
  more rules do exist. Also, it is very important the way they are
  combined together 

\end{frame}

\begin{frame}
  \frametitle{Trivial Simplifications}

  The following are trivial consequences of the semantic of \bitvectors and
  \formulae in general
  \vfill
  \pause
  \bitvectors trivial simplifications
  \begin{itemize}
    \item $t = t\ \Rightarrow\ \top$\ \ \ \ for a generic term
    \item $c = d\ \Rightarrow\ \bot$\ \ \ \ for two different constants $c$ and $d$
    \item $t \band 0\ldots0\ \Rightarrow\ 0\ldots0$\ \ \ \ for a generic term
    \item \ldots
  \end{itemize}
  \vfill
  \pause
  \begin{itemize}
    \item $\varphi \wedge \varphi\ \Rightarrow\ \varphi$\ \ \ \ for a generic formula
    \item $\varphi \wedge \top\ \Rightarrow\ \varphi$\ \ \ \ for a generic formula
    \item $\varphi \vee \top\ \Rightarrow\ \top$\ \ \ \ for a generic formula
    \item \ldots
  \end{itemize}

\end{frame}

\begin{frame}
  \frametitle{Ground Term evaluation}
  
  If a term is {\bf ground}, i.e., it contains no variables, then it
  can be always simplified to a single constant
  \vfill
  \pause
  Examples:
  \begin{itemize}
    \item $0000 :: 1000\ \Rightarrow 00001000$
    \item $0010[1:0]\ \Rightarrow 10$
    \item $0100 + 0101\ \Rightarrow 1001$
  \end{itemize}

\end{frame}

\begin{frame}
  \frametitle{Variable Elimination Rule}

  Suppose that the input formula $\varphi$ is of the kind
  \vfill
  $$
  \varphi'\ \wedge\ (\w{x}{n} = \w{t}{n})
  $$
  \vfill
  where $\w{x}{n}$ is a variable, and $\w{t}{n}$ is a term {\bf not containing} $\w{x}{n}$
  \pause
  \vfill
  then we can rewrite $\varphi$ as
  $$
  \varphi'[\w{t}{n}/\w{x}{n}]
  $$
  \vfill
  i.e., we replace every occurrence of $\w{x}{n}$ by $\w{t}{n}$. \pause We save $n$
  Boolean variables in the reduction to SAT

\end{frame}

\begin{frame}
  \frametitle{Concatenation Elimination Rule}

  Suppose that we have the equality
  \vfill
  $$
  \w{t}{n} :: \w{s}{m} = \w{r}{n} :: \w{u}{m}
  $$
  \vfill
  then, because the concatenations {\bf match}, we can rewrite it as
  \pause
  \vfill
  $$
  (\w{t}{n} = \w{r}{n}) \wedge (\w{s}{m} = \w{u}{m}) 
  $$
  \vfill
  this rewriting may give more opportunity for applications of previous rules

\end{frame}

\begin{frame}
  \frametitle{Exercizes}

  \begin{enumerate}

    \item Complete the missing cases in procedures Bit-Blast-Term and Bit-Blast
    \vfill
    \item Bit-Blast the formula $\neg(\w{x}{3} = 000) \wedge (\w{x}{3} \band \w{y}{3}) = (\w{x}{3} + \w{y}{3})$ 
    \vfill
    \item Simplify the formula $(\w{x}{4} :: \w{y}{4}) = (\w{z}{4} :: \w{x}{4}) \wedge \neg( \w{y}{4} = \w{z}{4})$ 

  \end{enumerate}

\end{frame}
