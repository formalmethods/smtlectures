\subsection{Solving}

\begin{frame}
  \frametitle{The Tableau}

  \scriptsize

  The equations $A\vec{x} = \vec{0}$ are kept in a {\bf tableau}, the most 
  important structure of the Simplex
  \vfill
  The variables are {\bf partitioned} into the set of {\bf non-basic} \nonbas
  and {\bf basic} \bas  variables 
  \vfill
  E.g., \bas = $\{ x_1, x_3, x_4 \}$, \nonbas = $\{ x_2, x_5, x_6 \}$
  $$
  \begin{array}{rcl}
    x_1 & = & 4 x_2 + x_5 \\
    x_3 & = & 5 x_2 + 3 x_6 \\
    x_4 & = & x_5 - x_6
  \end{array}
  $$
  \vfill
  non-basic variables can be considered as {\bf independent}, while basic variables
  assume values forced by the non-basic ones. E.g., in the row
  $$
    x_1 = 4 x_2 + x_5 
  $$
  suppose that $x_2 = 2, x_5 = 1$, then we set $x_1 = 9$

\end{frame}

\begin{frame}
  \frametitle{Solving}

  \scriptsize

  The \tsolver stores 
  \begin{itemize}
    \item the Tableau (does not grow/shrink)
    \item the active bounds on variables (initially none)
    \item the current model $\mu$ (initially all $0$, but could be chosen differently)
  \end{itemize}

  \vfill

  \begin{columns}

  \begin{column}{.4\textwidth}
  \begin{center}
  Tableau
  \end{center}
  $$
  \begin{array}{rcl}
    x_1 & = & a_{11} x_{m+1} \ldots + a_{1n} x_n  \\ 
    x_2 & = & a_{21} x_{m+1} \ldots + a_{2n} x_n  \\ 
    & \ldots \\                             
    x_i & = & a_{i1} x_{m+1} \ldots + a_{in} x_n  \\ 
    & \ldots \\                             
    x_m & = & a_{m1} x_{m+1} \ldots + a_{mn} x_n \\
    \\
    \\
  \end{array}
  $$
  \end{column}

  \begin{column}{.3\textwidth}
  \begin{center}
  $lb$~~~~~~~Bounds~~~~~~~$ub$
  \end{center}
  $$
  \begin{array}{rcccl}
    - \infty & \leq & x_1 & \leq & \infty \\
    - \infty & \leq & x_2 & \leq & \infty \\
    & & \ldots & & \\
    - \infty & \leq & x_i & \leq & \infty \\
    & & \ldots & & \\
    - \infty & \leq & x_m & \leq & \infty \\
    & & \ldots & & \\
    - \infty & \leq & x_n & \leq & \infty \\
  \end{array}
  $$
  \end{column}

  \begin{column}{.3\textwidth}
  \begin{center}
  $\mu$
  \end{center}
  $$
  \begin{array}{rcl}
  x_1 & \mapsto & 0 \\
  x_2 & \mapsto & 0 \\
  & \ldots \\
  x_i & \mapsto & 0 \\
  & \ldots \\
  x_m & \mapsto & 0 \\
  & \ldots \\
  x_n & \mapsto & 0 
  \end{array}
  $$
  \end{column}

  \end{columns}

  \vfill

  The \tsolver is in a consistent state if the model $(i)$ respects the tableau 
  and $(ii)$ satisfies the bounds (for all $x$, $lb(x) \leq \mu(x) \leq ub(x)$)

\end{frame}

\begin{frame}
  \frametitle{Asserting a bound on a non-basic variable}

  \scriptsize

  Asserting a bound $x \leq c$ (resp. $x \geq c$), $x \in \nonbas$ may result in \pause
  \begin{itemize}
    \item unsatisfiability, if $c < lb(x)$ (resp. $c > ub(x)$) \pause
    \item nothing, if $c > ub(x)$ (resp. $c < lb(x)$) \pause
    \item bound tightening, model not affected if $\mu(x) \leq c$ (resp. $\mu(x) \geq c$) \pause
    \item bound tightening, model affected if $\mu(x) > c$ (resp. $\mu(x) < c$) \pause
  \end{itemize}
  \vfill
  The last case is ``problematic'' because we need to 
  \begin{enumerate}[$(i)$]
    \item adjust $\mu(x)$: $\mu(x)$ is set to $c$ 
    \item adjust the values of basic variables
  \end{enumerate}
  We assume we have a function $Update(x,c)$ that implements $(i)-(ii)$
  \vfill \pause
  \begin{columns}

  \begin{column}{.4\textwidth}
  \begin{center}
  Tableau
  \end{center}
  $$
  \begin{array}{rcl}
    x_1 & = & - x_3 + x_4  \\ 
    x_2 & = &   x_3 + x_4  \\ 
    \\
    \\
  \end{array}
  $$
  \end{column}

  \begin{column}{.3\textwidth}
  \begin{center}
  $lb$~~~~~~~Bounds~~~~~~~$ub$
  \end{center}
  $$
  \begin{array}{rcccl}
    - \infty & \leq & x_1 & \leq & \infty \\
    - \infty & \leq & x_2 & \leq & \infty \\
    \only<1-9|handout:0>{-\infty}\only<10->{-8} & \leq & x_3 & \leq & \only<1-7|handout:0>{\infty}\only<8->{-4} \\
    - \infty & \leq & x_4 & \leq & \infty 
  \end{array}
  $$
  \end{column}

  \begin{column}{.3\textwidth}
  \begin{center}
  $\mu$
  \end{center}
  $$
  \begin{array}{rcl}
  x_1 & \mapsto & \only<1-8|handout:0>{0}\only<9->{4} \\
  x_2 & \mapsto & \only<1-8|handout:0>{0}\only<9->{-4} \\
  x_3 & \mapsto & \only<1-8|handout:0>{\coloneat{0}{8}}\only<9->{-4} \\
  x_4 & \mapsto & 0 
  \end{array}
  $$
  \end{column}

  \end{columns}
  \vfill
  \tsolver stack: \\
  \onslide<8->{$x_3 \leq -4$\ \ \ \ (tighten $ub(x_3)$, affects other values)\\}
  \onslide<10->{$x_3 \geq -8$\ \ \ \ (tighten $lb(x_3)$, does not affect other values)\\}
  \onslide<11->{$x_3 \leq 0$\ \ \ \ \ \  (does not tighten $ub(x_3)$)\\}

\end{frame}

\begin{frame}
  \frametitle{Asserting a bound on a basic variable}

  \scriptsize

  Asserting a bound $x \leq c$ (resp. $x \geq c$), $x \in \bas$ may result in
  the same $4$ cases as before
  \begin{itemize}
    \item unsatisfiability, if $c < lb(x)$ (resp. $c > ub(x)$) 
    \item nothing, if $c > ub(x)$ (resp. $c < lb(x)$) 
    \item bound tightening, model not affected if $\mu(x) \leq c$ (resp. $\mu(x) \geq c$)
    \item bound tightening, model affected if $\mu(x) > c$ (resp. $\mu(x) < c$) 
  \end{itemize}
  \vfill
  however, since a basic variable is {\bf dependent} from non-basic variables in the
  tableau, we cannot use function $Update$ directly. Before we need to 
  turn $x$ into a non-basic variables.
  \pause 
  \vfill
  So if $\mu(x) > c$ or $\mu(x) < c$ we have to
  \begin{enumerate}[$(i)$]
    \item {\bf turn $x$ into non-basic} (another non-basic variable will become basic instead)
    \item adjust $\mu(x)$: $\mu(x)$ is set to $c$ 
    \item adjust the values of basic variables
  \end{enumerate}
  \vfill
  Step $(i)$ is performed by a function $Pivot(x,y)$ (see next slide). Therefore asserting
  a bound on a basic variable $x$ consists in executing $Pivot(x,y)$ and then $Update(x,c)$
\end{frame}

\begin{frame}
  \frametitle{Pivoting}

  \scriptsize

  $Pivot(x,y)$ is the operation of swapping a basic variable $x$ with a non-basic variable $y$ 
  (how to choose $y$ ? See next slide)
  \vfill
  It consists of the following steps:
  \begin{enumerate}
    \item Take the row $x = a y + R$ in the tableau ($R = \mbox{rest of the polynome}$)
    \item Rewrite it as $y = \frac{x - R}{a}$
    \item Substitute $y$ with $\frac{x - R}{a}$ in all the other rows of the tableau (and simplify)
  \end{enumerate}
  \vfill\pause
  Example for $Pivot(x_1,x_3)$
  \vfill
  \ra{1.5}
  \begin{columns}

    \begin{column}{.33\textwidth}
      \begin{center}
      Step 1 
      \end{center}
      $$
      \begin{array}{rcl}
	\colone{x_1} & \colone{=} & \colone{3 x_2 + 4 x_3 - 5 x_4} \\
	x_5 & = & - x_2 - x_3 \\
	x_6 & = & 10 x_2 + 5 x_4	
      \end{array}
      $$
    \end{column}

    \begin{column}{.33\textwidth}
      \begin{center}
      Step 2 
      \end{center}
      $$
      \begin{array}{rcl}
	\colone{x_3} & \colone{=} & \colone{-\frac{3}{4} x_2 + \frac{1}{4} x_1 + \frac{5}{4} x_4} \\
	x_5 & = & - x_2 - x_3 \\
	x_6 & = & 10 x_2 + 5 x_4	
      \end{array}
      $$
    \end{column}

    \begin{column}{.33\textwidth}
      \begin{center}
      Step 3 
      \end{center}
      $$
      \begin{array}{rcl}
	x_3 & = & -\frac{3}{4} x_2 + \frac{1}{4} x_1 + \frac{5}{4} x_4 \\
	\colone{x_5} & \colone{=} & \colone{- \frac{1}{4} x_2 -\frac{1}{4} x_1 - \frac{5}{4} x_4} \\
	x_6 & = & 10 x_2 + 5 x_4	
      \end{array}
      $$
    \end{column}

  \end{columns}
  \vfill\pause
  we moved from $\bas = \{ x_1, x_5, x_6 \}$ to $\bas = \{ x_3, x_5, x_6 \}$ 

\end{frame}

\begin{frame}
  \frametitle{Choosing Pivoting Variable}

  \scriptsize

  Consider the following situation 
  \vfill
  \begin{columns}

  \begin{column}{.4\textwidth}
  \begin{center}
  Tableau
  \end{center}
  $$
  \begin{array}{rcl}
    & \ldots \\                             
    x_1 & = & 3 x_2 - 4 x_3 + 2 x_4 - x_5 \\ 
    & \ldots \\                             
    \\
    \\
  \end{array}
  $$
  \end{column}

  \begin{column}{.3\textwidth}
  \begin{center}
  $lb$~~~~~~~Bounds~~~~~~~$ub$
  \end{center}
  $$
  \begin{array}{rcccl}
     -4 & \leq & x_1 & \leq & 10 \\
      1 & \leq & x_2 & \leq & 3 \\
     -4 & \leq & x_3 & \leq & -1 \\
      1 & \leq & x_4 & \leq & 2 \\
     -1 & \leq & x_5 & \leq & 10 \\
  \end{array}
  $$
  \end{column}

  \begin{column}{.3\textwidth}
  \begin{center}
  $\mu$
  \end{center}
  $$
  \begin{array}{rcl}
  x_1 & \mapsto & \colone{12} \\
  x_2 & \mapsto & 1 \\
  x_3 & \mapsto & -1 \\
  x_4 & \mapsto & 2 \\
  x_5 & \mapsto & -1 
  \end{array}
  $$
  \end{column}

  \end{columns}
  \vfill
  which among $\nonbas = \{ x_2, x_3, x_4 \}$ do I choose for pivoting ? Clearly, the value of $\mu(x_1)$
  is too high, I have to decrease it by playing with the values of \nonbas: \pause
  \begin{itemize}
    \item $  3 x_2$ cannot decrease, as $\mu(x_2) = lb(x_2)$ and cannot be moved down \pause
    \item $- 4 x_3$ cannot decrease, as $\mu(x_3) = ub(x_3)$ and cannot be moved up \pause
    \item $  2 x_4$ can decrease, as  $\mu(x_4) = ub(x_4)$, and can be moved down \pause
    \item $-   x_5$ can decrease, as  $\mu(x_5) = lb(x_5)$, and can be moved up
  \end{itemize}
  \vfill\pause
  both $x_4$ and $x_5$ are therefore good candidates for pivoting. To avoid loops, 
  choose variable with smallest subscript (Bland's Rule). This rule is not necessarily
  efficient, though

\end{frame}

\begin{frame}
  \frametitle{Detect Unsatisfiability}

  \scriptsize

  There might be cases in which {\bf no suitable variable for pivoting can be found}.
  This indicates unsatisfiability. \pause
  Consider the following where we have just asserted $x_1 \leq 9$
  \vfill
  \begin{columns}

  \begin{column}{.4\textwidth}
  \begin{center}
  Tableau
  \end{center}
  $$
  \begin{array}{rcl}
    & \ldots \\                             
    x_1 & = & 3 x_2 - 4 x_3 + 2 x_4 - x_5 \\ 
    & \ldots \\                             
    \\
    \\
  \end{array}
  $$
  \end{column}

  \begin{column}{.3\textwidth}
  \begin{center}
  $lb$~~~~~~~Bounds~~~~~~~$ub$
  \end{center}
  $$
  \begin{array}{rcccl}
     -4 & \leq & x_1 & \leq & 9 \\
      1 & \leq & x_2 & \leq & 3 \\
     -4 & \leq & x_3 & \leq & -1 \\
      2 & \leq & x_4 & \leq & 2 \\
     -1 & \leq & x_5 & \leq & -1 \\
  \end{array}
  $$
  \end{column}

  \begin{column}{.3\textwidth}
  \begin{center}
  $\mu$
  \end{center}
  $$
  \begin{array}{rcl}
  x_1 & \mapsto & \colone{12} \\
  x_2 & \mapsto & 1 \\
  x_3 & \mapsto & -1 \\
  x_4 & \mapsto & 2 \\
  x_5 & \mapsto & -1 
  \end{array}
  $$
  \end{column}

  \end{columns}
  \vfill
  no variable among $\nonbas = \{ x_2, x_3, x_4 \}$ can be chosen for pivoting. This is because (due to tableau)
  $$x_2 \geq 3\ \swedge\ x_3 \leq -1\ \swedge\ x_4 \geq 4\ \swedge\ x_5 \leq -1\ \ \Rightarrow\ \ x_1 \geq 12\ \ \Rightarrow\ \ \neg (x_1 \leq 9)$$
  \vfill
  Therefore
  $$\{ x_2 \geq 3,\ x_3 \leq -1,\ x_4 \geq 4,\ x_5 \leq -1,\ \neg( x_1 \leq 9 ) \}$$
  is a \tconflict (modulo the tableau)

\end{frame}

\begin{frame}
  \frametitle{Solving}

  \scriptsize

  \begin{tabbing}
  asv \= a \= a \= a \= as \= asdfasdfasdfasdfasdfasdfasdfasdf \= \kill
  1  \> while( $true$ ) \\
     \> \\
  2  \> \> pick first $x_i \in \bas$ such that $\mu(x_i) < lb(x_i)$ or $\mu(x_i) > ub(x_i)$ \\
  3  \> \> if ( there is no such $x_i$ ) return $sat$ \\
     \> \\
  4  \> \> $x_j = ChoosePivot( x_i )$ \\
  5  \> \> if ( $x_j$ == $undef$ ) return $unsat$ \\ 
  6  \> \> $Pivot(x_i,x_j)$ \\
     \> \\
  7  \> \> if ( $\mu(x_i) < lb(x_i)$ ) \\
  8  \> \> \> $Update( x_i, lb(x_i) )$ \\
     \> \\
  9  \> \> if ( $\mu(x_i) > ub(x_i)$ ) \\
  10 \> \> \> $Update( x_i, ub(x_i) )$ \\
     \> \\
  11 \> end
  \end{tabbing}

  $x_j = ChoosePivot( x_i )$: returns variable to use for pivoting with $x_i$, or $undef$ if conflict\\
  pick first $x_i$ \ldots: it's again Bland's rule

\end{frame}
