\documentclass[xcolor=dvipsnames
              ,handout
              ]{beamer} 

\usetheme{Madrid} 
%\setbeamertemplate{blocks}[shadow=false] 
\setbeamertemplate{navigation symbols}{} 
\setbeamertemplate{items}[square]
\setbeamertemplate{sections/subsections in toc}[square]

\definecolor{myblueend}{rgb}{0.058,0.132,0.42}
\definecolor{mybluemiddle}{rgb}{0.31,0.45,0.64}
\definecolor{mybluestart}{rgb}{0.17,0.28,0.48}
\definecolor{mygreen}{rgb}{0,0.7,0}
\definecolor{mylightgreen}{rgb}{0.7,1,0.7}
\definecolor{mylightblue}{rgb}{0.7,0.7,1}
\definecolor{mylightblack}{rgb}{0.7,0.7,0.7}
\definecolor{mylightred}{rgb}{1,0.7,0.7}
\definecolor{oproverblue}{RGB}{12,83,144}
\definecolor{oproveryellow}{RGB}{255,156,55}
\definecolor{mydarkgreen}{rgb}{0.17,0.48,0.28}
\definecolor{mydarkred}{rgb}{0.48,0.28,0.17}
\definecolor{grey}{rgb}{0.7,0.7,0.7}
\definecolor{darkgrey}{rgb}{0.5,0.5,0.5}

%\setbeamercolor{palette primary}{fg=white,bg=mybluestart}
%\setbeamercolor{palette secondary}{fg=white,bg=mybluestart}
%\setbeamercolor{palette tertiary}{fg=white,bg=mybluestart}
%\setbeamercolor{palette quaternary}{fg=white,bg=mybluestart}
\setbeamercolor{palette primary}{fg=white,bg=oproverblue}
\setbeamercolor{palette secondary}{fg=white,bg=oproverblue}
\setbeamercolor{palette tertiary}{fg=white,bg=oproverblue}
\setbeamercolor{palette quaternary}{fg=white,bg=oproverblue}
\setbeamercolor{titlelike}{parent=palette quaternary}

\setbeamercolor{item}{fg=oproverblue}
\setbeamercolor{block title}{fg=white,bg=oproverblue}
\setbeamercolor{block title example}{fg=white,bg=oproveryellow}
%\setbeamercolor{block title alert}{fg=white,bg=mydarkred}

\usefonttheme{serif}

%%
%% TOOLS
%% 
\newcommand{\opensmt}{{\sc OpenSMT}\xspace}
\newcommand{\yices}{{\sc Yices}\xspace}
\newcommand{\mathsat}{{\sc MathSAT}\xspace}
\newcommand{\cvcfour}{{\sc CVC4}\xspace}
\newcommand{\zthree}{{\sc Z3}\xspace}
\newcommand{\boolector}{{\sc Boolector}\xspace}
\newcommand{\verit}{{\sc veriT}\xspace}
\newcommand{\stp}{{\sc STP}\xspace}
\newcommand{\minisat}{{\sc MiniSAT}\xspace}
%%
%% BOOLEAN OPERATORS
%%
\newcommand{\swedge}{\,\wedge\,}
\newcommand{\svee}{\,\vee\,}
\newcommand{\impl}{\,\rightarrow\,}
%%
%% SMTLIB LOGICS
%% 
\newcommand{\Idl}{\ensuremath{\mathcal{IDL}}\xspace}
\newcommand{\Rdl}{\ensuremath{\mathcal{RDL}}\xspace}
\newcommand{\Uf}{\ensuremath{\mathcal{UF}}\xspace}
\newcommand{\Lia}{\ensuremath{\mathcal{LIA}}\xspace}
\newcommand{\Lra}{\ensuremath{\mathcal{LRA}}\xspace}
\newcommand{\Arrays}{\ensuremath{\mathcal{A}}\xspace}
\newcommand{\Bitvectors}{\ensuremath{\mathcal{BV}}\xspace}
\newcommand{\T}{\ensuremath{\mathcal{T}}\xspace}
\newcommand{\B}{\ensuremath{\mathcal{B}}\xspace}
%%
%% SETS
%%
\newcommand{\Int}{\ensuremath{\mathbb{Z}}\xspace}
\newcommand{\Rat}{\ensuremath{\mathbb{Q}}\xspace}
\newcommand{\Rea}{\ensuremath{\mathbb{R}}\xspace}
\newcommand{\Boo}{\ensuremath{\mathbb{B}}\xspace}
%%
%% SORTS
%%
\newcommand{\SInt}{{\tt Int}\xspace}
\newcommand{\SRea}{{\tt Real}\xspace}
\newcommand{\SBoo}{{\tt Bool}\xspace}
\newcommand{\SBv}[1]{{\tt BV}$_{[#1]}$\xspace}
%%
%% SMT specific
%%
\newcommand{\tconflict}{\T-conflict\xspace}
\newcommand{\tconflicts}{\T-conflicts\xspace}
\newcommand{\tterm}{\T-term\xspace}
\newcommand{\tterms}{\T-terms\xspace}
\newcommand{\tatom}{\T-atom\xspace}
\newcommand{\tatoms}{\T-atoms\xspace}
\newcommand{\tlit}{\T-literal\xspace}
\newcommand{\tlits}{\T-literals\xspace}
\newcommand{\tformula}{\T-formula\xspace}
\newcommand{\batom}{\B-atom\xspace}
\newcommand{\batoms}{\B-atoms\xspace}
\newcommand{\blit}{\B-literal\xspace}
\newcommand{\blits}{\B-literals\xspace}
\newcommand{\babst}[1]{#1^{\B}}
\newcommand{\tsolver}{\T-solver\xspace}
\newcommand{\tsolvers}{\T-solvers\xspace}
%%
%% SAT specific
%%
\newcommand{\dec}[2]{\stackrel{\textcolor{oproveryellow}{#2}}{#1}}
%%
%% BIT-VECTORS
%%
\newcommand{\w}[2]{\ensuremath{#1_{[#2]}}}
\newcommand{\band}{\,{\bf AND}\,}
\newcommand{\bor}{\,{\bf OR}\,}
\newcommand{\bnot}{{\bf NOT}\,}
\newcommand{\bitandsymb}{\,\, {\bf AND}}
\newcommand{\bit}[2]{#1\ensuremath{^#2}}
%%
%% IDL graphs
%%
\newcommand{\idlnode}[2]{\frac{#1}{#2}}
%%
%% LRA Solver
%%
\newcommand{\bas}{\ensuremath{\mathcal{B}}\xspace}
\newcommand{\nonbas}{\ensuremath{\mathcal{N}}\xspace}
%%
%% MISC
%%
\newcommand{\Lbrack}{\ensuremath{[\mspace{-3mu}[}}
\newcommand{\Rbrack}{\ensuremath{]\mspace{-3mu}]}}
\newcommand{\inter}[1]{\ensuremath{\Lbrack #1 \Rbrack}}
\newcommand{\COMMENT}[1]{}
\newcommand{\hl}[1]{\colorbox{oproveryellow}{\bf #1}}
\newcommand{\colfou}[1]{\textcolor{grey}{#1}}
\newcommand{\formulae}{formul\ae\xspace}
\newcommand{\smtsolvers}{SMT-solvers\xspace}
\newcommand{\smtsolver}{SMT-solver\xspace}
\newcommand{\satsolvers}{SAT-solvers\xspace}
\newcommand{\satsolver}{SAT-solver\xspace}
\newcommand{\bitvectors}{Bit-Vectors\xspace}
\newcommand{\bitvector}{Bit-Vector\xspace}
\newcommand{\colone}[1]{\textcolor{red}{#1}}
\newcommand{\coltwo}[1]{\textcolor{mygreen}{#1}}
\newcommand{\coloneat}[2]{\textcolor<#2>{red}{#1}}
\newcommand{\coltwoat}[2]{\textcolor<#2>{mygreen}{#1}}
\newcommand{\colfouat}[2]{\textcolor<#2>{grey}{#1}}
\newcommand{\claset}{\mathcal{C}}
\newcommand{\ra}[1]{\renewcommand{\arraystretch}{#1}}

\usepackage{epsfig}
\usepackage{graphicx}
\usepackage{color}
\usepackage{amstext}
\usepackage{amssymb}
\usepackage{amsfonts}
\usepackage{amsmath}
\usepackage{amsthm}
\usepackage{xspace}
\usepackage{multirow}
\usepackage{tabularx,colortbl}
\usepackage{alltt}
\usepackage{bussproofs}
\usepackage{algorithm2e}
\usepackage{datetime}
\usepackage{tikz}


\title[Overview]{Satisfiability Modulo Theories\\ Lezione 1 - Overview \\ {\tiny (slides revision: \today, \currenttime)}}
\author[R. Bruttomesso]{\large Roberto Bruttomesso}
\date{20 Ottobre 2011}
\institute[SMT]{\large Seminario di Logica Matematica \\(Corso Prof. Silvio Ghilardi)}
\logo{ \vspace{-6pt} \includegraphics[scale=0.15]{imgs/logo-ita.png} }

\begin{document}

\frame{\titlepage}

\begin{frame}[fragile]
  \frametitle{FAQ}

Le slides seguono la dispensa (in fase di scrittura !)
\begin{center}
``Satisfiability Modulo Theories''
\end{center}
e sono entrambi disponibili da 
\url{http://www.oprover.org/roberto/teaching/smt}, dove
trovate anche i puntatori ai tool e agli esempi usati qui

\vfill
Per chi cercasse un libro su questi argomenti, forse quello che
si avvicina di piu' e'
\begin{verbatim}
Decision Procedures - An Algorithmic Point of View
\end{verbatim}
(\url{www.decision-procedures.org})

\vfill
Per ricevimento la mia stanza e' la S206, II piano, via Comelico,
previa richiesta via email \url{roberto.bruttomesso@gmail.com}

\end{frame}

\begin{frame}
  \frametitle{Outline}
  \tableofcontents
\end{frame}

\section{A gentle introduction to SMT}

\begin{frame}
  \frametitle{Introduction}

  \bitvectors are extremely useful data structures,
  used to symbolically represent hardware and software
  constructs (see later)
  \vfill
  \pause
  The world of \bitvectors is a {\bf finite} world, i.e.,
  with \bitvectors it is not possible to represent/handle 
  arbitrarily large numbers 
  \vfill
  \pause
  Indeed, when speaking about \bitvectors we always
  associate a {\bf width}  
  (which is usually a power of 2, often 32 or 64)
  \vfill
  \pause
  The width specifies the (maximum) {\bf number of bits} used to
  represent variables and terms
  \vfill
  \pause
  \bitvector \formulae are mathematically characterized by 
  the theory of \bitvectors \Bitvectors
  
\end{frame}

\subsection{Syntax}

\begin{frame}
  \frametitle{Bit-Vectors}
  
  A bit-vector is an array of bits 
  \smallskip \\
  \begin{center}
  \scalebox{.3}{\input{bv_ab.pdf_t}}
  \end{center} 
  \vfill
  \pause

  Selection (or Extraction): $\w{a}{3}[1:0]$ 
  \smallskip \\
  \begin{center}
  \scalebox{.3}{\input{bv_a_sel.pdf_t}} 
  \end{center}

  \vfill
  \pause

  Notice that
  \begin{itemize}
    \item $\w{a}{n}[i:j]$ returns a \bitvector of width $i - j + 1$ ($0 \leq j \leq i \leq n - 1$) \pause
    \item $\w{a}{n}[n-1:0]$ \pause $= \w{a}{n}$ \pause
    \item Selection has precedence over any other operator
  \end{itemize}

\end{frame}

\begin{frame}
  \frametitle{Bit-Vectors}
  
  A bit-vector is an array of bits 
  \smallskip \\
  \begin{center}
  \scalebox{.3}{\input{bv_ab.pdf_t}}
  \end{center} 
  \vfill

  Concatenation $\w{a}{3} :: \w{b}{3}$ 
  \smallskip \\
  \begin{center}
  \scalebox{.3}{\input{bv_ab_conc.pdf_t}} 
  \end{center}

  \vfill
  \pause

  Notice that
  \begin{itemize}
    \item $\w{a}{n} :: \w{b}{m}$ returns a \bitvector of width $n+m$ \pause
    \item $\w{a}{n}[n-1:i] :: \w{a}{n}[i-1:0]$ \pause $= \w{a}{n}[n-1:0]$ \pause $= \w{a}{n}$
  \end{itemize}

\end{frame}

\begin{frame}
  \frametitle{Bit-Vectors}
  
  A bit-vector is an array of bits 
  \smallskip \\
  \begin{center}
  \scalebox{.3}{\input{bv_ab.pdf_t}}
  \end{center} 
  \vfill

  Arithmetic $\w{a}{3} + \w{b}{3}$ 
  \smallskip \\
  \begin{center}
  \scalebox{.3}{\input{bv_ab_plus.pdf_t}} 
  \end{center}

  \vfill
  \pause

  Notice that
  \begin{itemize}
    \item To be precise, we should have written $\w{a}{3} +_{[3]} \w{b}{3}$ (widths must be the same)
    \item Semantic is that of {\bf modular} arithmetic
  \end{itemize}

\end{frame}

\begin{frame}
  \frametitle{Bit-Vectors}
  
  A bit-vector is an array of bits 
  \smallskip \\
  \begin{center}
  \scalebox{.3}{\input{bv_ab.pdf_t}}
  \end{center} 
  \vfill

  Bitwise $\w{a}{3} \band \w{b}{3}$ 
  \smallskip \\
  \begin{center}
  \scalebox{.3}{\input{bv_ab_bitw.pdf_t}} 
  \end{center}

  \vfill
  \pause

  Notice that
  \begin{itemize}
    \item Again, to be precise, we should have written $\w{a}{3} \band_{[3]} \w{b}{3}$ (widths must be the same)
    \item Used to compute bit-mask operations
  \end{itemize}

\end{frame}

\begin{frame}
  \frametitle{A (non-exhaustive) list of operators and predicates}

  Each \bitvector term of width $n$, is associated with a sort \SBv{n} ($n \geq 1$)
  \vfill
  \pause

  \begin{center}
  \begin{tabular}{|l|l|l|l|}
    \hline
    Name                 & Symb      & Type                        & Signature \\
    \hline               
    Selection            & $\_[i:j]$ & \multirow{2}{*}{Core}       & \SBv{n}                  $\rightarrow$ \SBv{i-j+1} \\
    Concatenation        & $::$      &                             & \SBv{n} $\times$ \SBv{m} $\rightarrow$ \SBv{n+m} \\ 
    \hline               
    Addition             & $+$       & \multirow{5}{*}{Arith.}     & \SBv{n} $\times$ \SBv{n} $\rightarrow$ \SBv{n} \\
    Subtraction          & $-$       &                             & \SBv{n} $\times$ \SBv{n} $\rightarrow$ \SBv{n} \\
    Multiplication       & $*$       &                             & \SBv{n} $\times$ \SBv{n} $\rightarrow$ \SBv{n} \\
    Less than (signed)   & $<_s$     &                             & \SBv{n} $\times$ \SBv{n} $\rightarrow$ \SBoo \\
    Less than (unsigned) & $<_u$     &                             & \SBv{n} $\times$ \SBv{n} $\rightarrow$ \SBoo \\
    \hline
    Bitwise and          & $\band$   & \multirow{3}{*}{Bitwise}    & \SBv{n} $\times$ \SBv{n} $\rightarrow$ \SBv{n} \\
    Bitwise or           & $\bor$    &                             & \SBv{n} $\times$ \SBv{n} $\rightarrow$ \SBv{n} \\
    Bitwise not          & $\bnot\_$ &                             & \SBv{n}                  $\rightarrow$ \SBv{n} \\
    \hline
  \end{tabular}
  \end{center}
  \vfill
  \pause

  Moreover, we have constants, e.g., $101101_{[6]}$

\end{frame}

\subsection{Semantic}

\begin{frame}
  \frametitle{\bitvector semantic}

  \scriptsize

  Each sort \SBv{n} is associated with a domain $D_n = \{ 0, 1, \ldots, 2^{n-1} \}$ \\ \pause
  For example \SBv{4} is associated with $D_{4} = \{ 0, 1, \ldots, 15 \}$
  \vfill
  \pause
  As usual, the semantic for the other terms depends on a particular {\bf assignment}
  to the variables
  \vfill
  \pause
  Each variable $\w{x}{n}$ is associated with a function \inter{\w{x}{n}} 
  of type $D_n \rightarrow \{ 0, 1 \}$
  \vfill
  \pause
  \scalebox{.7}{\input{semantic_example.pdf_t}}
  \vfill
  \pause
  $nat_n(\_)$ is a helper meta-function, to facilitate the presentation 

\end{frame}

\begin{frame}
  \frametitle{\bitvector semantic}

  \scriptsize
  $$
  \begin{array}{rcl}
    %% Constant
    \inter{\w{c}{n}} & := & \lambda x \in [0,n-1].\, 
    \left\{
    \begin{array}{ll}
      0 & \mbox{ if the } x\mbox{-th bit is } 0 \\
      1 & \mbox{ otherwise } 
    \end{array}
    \right. \\
    \medskip \\
    %% CONCATENATION
    \inter{\w{t}{l} :: \w{s}{k}} & := & \lambda x \in [0,\ldots,l+k-1].\,
    \left\{
    \begin{array}{ll}
      \inter{\w{s}{n}}(x) & \mbox{ if } x < l \\
      \inter{\w{t}{n}}(x-l) & \mbox{ otherwise } 
    \end{array}
    \right. \\
    \smallskip \\
    %% SELECTION
    \inter{\w{t}{n}[i:j]} & := & \lambda x \in [0,i-j+1].\, \inter{\w{t}{n}}(x+j) \\
    \medskip \\
    %% PLUS
    \inter{\w{t}{n} + \w{s}{n}} & := & nat_n^{-1}(nat_n(\inter{\w{t}{n}}) + nat_n(\inter{\w{s}{n}}))\ \%\ 2^n \\
    \medskip \\
    %% BITAND
    \inter{\w{t}{n} \band \w{s}{n}} & := & \lambda x \in [0,n-1].\,
    \left\{
    \begin{array}{ll}
    0 & \mbox{ if } \inter{\w{t}{n}}(x) = 0 \\
    0 & \mbox{ if } \inter{\w{s}{n}}(x) = 0 \\
    1 & \mbox{ otherwise } 
    \end{array} 
    \right. \\
    \medskip \\
    %% LESS THAN
    \inter{\w{t}{n} <_u \w{s}{n}} & := & 
    \left\{
    \begin{array}{ll}
    \top & \mbox{ if } nat_n(\inter{\w{t}{n}}) <_{u} 
                       nat_n(\inter{\w{s}{n}}) \\
    \bot & \mbox{ otherwise } 
    \end{array}
    \right.
  \end{array}
  $$

\end{frame}

\subsection{The Eager and the Lazy approaches}

\begin{frame}
  \frametitle{The Eager and Lazy Approaches}

  Recall that in our reduction to SAT we need to encode
  \begin{enumerate}[$(i)$]
    \item incompatible relations between \tatoms exhaustively 
    \item the structure of $\varphi$
  \end{enumerate} 
  \pause
  \vfill
  Condition $(ii)$ is easy to encode. The critical condition is $(i)$. 
  If we have 3 \tatoms $a, b, c$, then we need to check whether
  \begin{itemize}
    \item $a$ and $b$ are incompatible 
    \item $a$ and $\neg b$ are incompatible 
    \item \ldots
    \item $a$ and $b$ and $c$ are incompatible 
    \item \ldots
  \end{itemize} 
  \pause
  \vfill
  Potentially, this leads to checking $O(2^n)$ relations, if $n$
  \tatoms are in the formula

\end{frame}

\begin{frame}
  \frametitle{The Eager and Lazy Approaches}

  There are (at least) two ways to discover incompatibilities
  \begin{itemize}
    \item {\bf eagerly} adding them before calling a \satsolver (eager approach)
	  \begin{description}
	    \item[$+$] Easy to implement: \satsolver used as black-box
	    \item[$+$] Good for bit-vectors theories
	    \item[$-$] Potentially generates too big encoding: needs heuristics
	               to make it efficient
	    \item[$-$] Bad for arithmetic theories
	  \end{description} \pause
    \item {\bf lazily}, by adding them during the \satsolver's search (lazy approach)
	  \begin{description}
	    \item[$+$] Generates smaller encodings
	    \item[$+$] Good for arithmetic theories
	    \item[$+$] Modular approach: allows easy theory combination
	    \item[$-$] Trickier to implement: \satsolver has to be ``openen''
	  \end{description}
  \end{itemize}
  \vfill
  \pause
  Most of this course will be devoted to the lazy approach, 
  which is nowadays the most successful technique available

\end{frame}

\begin{frame}
  \frametitle{The Eager Approach}

  \begin{center}
  \scalebox{.5}{\input{eager.pdf_t}}
  \end{center}

\end{frame}

\begin{frame}
  \frametitle{The Lazy Approach}

  \begin{center}
  \scalebox{.5}{\section{Lecture 4 - The Lazy Approach}

\begin{frame}
  \frametitle{The Lazy Approach}

  \scriptsize

  Notice that
  \begin{itemize}
    \item Assignments $\mu$ of $\varphi$ are many (potentially $\infty$),
          infeasible to check if any of them is a model {\bf systematically}
    \item Models $\babst{\mu}$ of $\babst{\varphi}$ are finite in number,
          and easy to enumerate with a SAT-solver
    \item A model $\babst{\mu}$ is nothing but a {\bf conjunction of \tatoms},
          can be checked efficiently with a \tsolver
  \end{itemize}
  \vfill
  These observations suggest us a methodology
  to tackle the SMT(\T) problem
  \begin{itemize}
    \item Enumerate a Boolean model $\babst{\mu}$ of $\babst{\varphi}$ (abstraction). If no model 
	  exist we are done ($\varphi$ is unsatisfiable) 
    \item Check if $\babst{\mu}$ is satisfiable using the \tsolver. If so $\babst{\mu}$ can be extended 
          to a model $\mu$ of $\varphi$, and so we are done ! ($\varphi$ is satisfiable) 
    \item It not, we tell the SAT-solver not to enumerate $\babst{\mu}$ again,
          thus {\bf cutting away systematically an infinite number} 
	  of assignments for $\varphi$ (refinement) 
    \item It can be blocked by adding a clause $\neg \babst{\mu}$. Go up
    \item It terminates because there are finite Boolean models
  \end{itemize}

\end{frame}

\begin{frame}
  \frametitle{The Lazy Approach}

  \scriptsize
  
  The lazy approach falls into the so-called {\bf abstraction-refinement} 
  paradigm
  \vfill
  \begin{center}
  \scalebox{.5}{\subsection{Lazy Approach as Abstraction Refinement}

\begin{frame}
  \frametitle{Abstraction}
  Assigment relations
  \vfill
  \begin{center}
  \begin{tabular}{ccc}

    \begin{minipage}{.2\textwidth}

    $\babst{\varphi}$ \\
    \\
    \\
    \\
    \\
    \\
    $\varphi$ 

    \end{minipage}

    &

    \begin{minipage}{.55\textwidth}
      \begin{overlayarea}{\textwidth}{5cm}
	\only<1-3|handout:0>{\scalebox{.4}{\input{assignments_1.pdf_t}}}
	\only<4|handout:0>{\scalebox{.4}{\input{assignments_2.pdf_t}}}
	\only<5|handout:0>{\scalebox{.4}{\input{assignments_3.pdf_t}}}
	\only<6|handout:0>{\scalebox{.4}{\input{assignments_4.pdf_t}}}
	\only<7>{\scalebox{.4}{\input{assignments_5.pdf_t}}}
      \end{overlayarea}
    \end{minipage}

    &

    \begin{minipage}{.3\textwidth}

    \onslide<3->{$2^n$} \\
    \onslide<3->{($n = |\{ a_i \}|$)}\\
    \\
    \\
    \\
    \onslide<2->{$\infty$} \\ 
    \onslide<2->(can be) 

    \end{minipage}

  \end{tabular}
  \end{center}

\end{frame}

\begin{frame}
  \frametitle{Abstraction}
  Model relations
  \begin{itemize}
    \item<2-> if $\mu$ is a model for $\varphi$, then $\babst{\mu}$ is a model for $\babst{\varphi}$
    \item<3-> if $\babst{\mu}$ is not a model for $\babst{\varphi}$, then there is no $\mu$ that is a model for $\varphi$
    \item<4-> there may be some model $\babst{\mu}$ for $\babst{\varphi}$ that does not map to any model 
	      $\mu$ for $\varphi$
  \end{itemize}
  \vfill
  \begin{center}
  \begin{tabular}{cc}

    \begin{minipage}{.2\textwidth}

    $\babst{\varphi}$ \\
    \\
    \\
    \\
    \\
    \\
    $\varphi$ 

    \end{minipage}

    &

    \begin{minipage}{.7\textwidth}
      \begin{overlayarea}{\textwidth}{5cm}
	\only<1|handout:0>{\scalebox{.4}{\input{models_1.pdf_t}}}
	\only<2|handout:0>{\scalebox{.4}{\input{models_2.pdf_t}}}
	\only<3|handout:0>{\scalebox{.4}{\input{models_3.pdf_t}}}
	\only<4>{\scalebox{.4}{\input{models_4.pdf_t}}}
      \end{overlayarea}
    \end{minipage}

  \end{tabular}
  \end{center}

\end{frame}

\begin{frame}
  \frametitle{Abstraction Refinement}

  \scriptsize

  Notice that
  \begin{itemize}
    \item Assignments $\mu$ of $\varphi$ are many (potentially $\infty$),
          infeasible to check if any of them is a model {\bf systematically}
    \item Models $\babst{\mu}$ of $\babst{\varphi}$ are finite in number,
          and easy to enumerate with a SAT-solver
    \item A model $\babst{\mu}$ is nothing but a {\bf conjunction of \tatoms},
          can be checked efficiently with a \tsolver
  \end{itemize}
  \vfill
  \pause
  These observations suggest us a methodology
  to tackle the SMT(\T) problem
  \begin{itemize}
    \item Enumerate a Boolean model $\babst{\mu}$ of $\babst{\varphi}$ (abstraction). If no model 
	  exist we are done ($\varphi$ is unsatisfiable) \pause
    \item Check if $\babst{\mu}$ is satisfiable using the \tsolver. If so $\babst{\mu}$ can be extended 
          to a model $\mu$ of $\varphi$, and so we are done ! ($\varphi$ is satisfiable) \pause
    \item It not, we tell the SAT-solver not to enumerate $\babst{\mu}$ again,
          thus {\bf cutting away systematically an infinite number} 
	  of assignments for $\varphi$ (refinement) \pause
    \item It can be blocked by adding a clause $\neg \babst{\mu}$. Go up \pause
    \item It terminates because there are finite Boolean models
  \end{itemize}

\end{frame}

\begin{frame}
  \frametitle{Abstraction Refinement}

  \scriptsize
  
  The lazy approach falls into the so-called {\bf abstraction-refinement} 
  paradigm
  \vfill
  \begin{center}
  \scalebox{.5}{\input{ar.pdf_t}}
  \end{center}

\end{frame}
}
  \end{center}

\end{frame}

\begin{frame}
  \frametitle{The Lazy Approach}

  \scriptsize

  The interaction described naturally falls within the
  CDCL style, enriched with a \tsolver
      $$\varphi \equiv 
      (x=3 \vee \neg (x<3))\ \swedge\
      (x=3 \vee \neg (x>3))\ \swedge\
      (x>3 \vee \neg (x<3))\ \swedge\
      (x>3 \vee \neg (x=3))$$ 
  \vfill
  \begin{columns}

    \begin{column}{.5cm}
      \vspace{-165pt}
      $\babst{\varphi} \equiv$
    \end{column}

    \begin{column}{5cm}
      $(\coltwoat{\coloneat{a_1}{8-13}}{2-6} \vee \coltwoat{\neg a_2}{9-13})$ \\
      $(\coltwoat{\coloneat{a_1}{8-13}}{2-6} \vee \coltwoat{\coloneat{\neg a_3}{3-6}}{10-13})$ \\
      $(\coloneat{\coltwoat{a_3}{3-6}}{10-13} \vee \coltwoat{\neg a_2}{9-13})$ \\
      $(\coloneat{\coltwoat{a_3}{3-6}}{10-13} \vee \coloneat{\coltwoat{\neg a_1}{8-13}}{2-6})$ \\
      \onslide<6->{$(\coloneat{\coltwoat{\neg a_1}{10-13}}{6} \vee \coltwoat{\coloneat{\neg a_3}{6}}{10-13})$ \\}
      \onslide<7->{$(\coltwoat{\neg a_1}{8-13})$ \\}
      \onslide<13->{$(\coloneat{a_1}{13} \vee \coloneat{a_2}{13} \vee \coloneat{a_3}{13})$ \\}
      \onslide<14->{$(\ )$ \\}
      \bigskip
      $a_1 \equiv x=3$ \\
      $a_2 \equiv x<3$ \\
      $a_3 \equiv x>3$ \\
      \bigskip
      $\babst{\mu}$: $\{\ \only<2-6|handout:0>{a_1}\only<3-6|handout:0>{, a_3}\only<8-13|handout:0>{\neg a_1}\only<9-13|handout:0>{, \neg a_2}\only<10-13|handout:0>{, \neg a_3}\ \}$ \\
      \bigskip
      \begin{tabular}{rl}
      SAT-solver: & \only<1,4-5,11-12|handout:0>{Idle}\only<2|handout:0>{Decision}\only<3,8-10|handout:0>{BCP}\only<6,13|handout:0>{Learn}\only<7,14|handout:0>{Conf. Analysis, Backtrack}\only<15>{UNS} \\
        \tsolver: & \only<1,2,3,6-10,13->{Idle}\only<4,5,11,12|handout:0>{Is $\babst{\mu}$ \T-satisfiable ?}\only<5,12|handout:0>{ NO}
      \end{tabular}
    \end{column}

    \begin{column}{5cm}
      \begin{overlayarea}{5cm}{5cm}
	\only<1|handout:0>{\scalebox{.6}{\input{search_0.pdf_t}}}
	\only<2|handout:0>{\scalebox{.6}{\input{search_1.pdf_t}}}
	\only<3,4,5|handout:0>{\scalebox{.6}{\input{search_2.pdf_t}}}
	\only<6,7|handout:0>{\scalebox{.6}{\input{search_3.pdf_t}}}
	\only<8|handout:0>{\scalebox{.6}{\input{search_4.pdf_t}}}
	\only<9|handout:0>{\scalebox{.6}{\input{search_5.pdf_t}}}
	\only<10-12|handout:0>{\scalebox{.6}{\input{search_6.pdf_t}}}
	\only<13->{\scalebox{.6}{\input{search_7.pdf_t}}}
      \end{overlayarea}
    \end{column}

  \end{columns}

\end{frame}

\begin{frame}
  \frametitle{Lecture 4 - Exercize 1}

  \scriptsize

  \begin{tabular}{ccc}
    \begin{minipage}{.4\textwidth}
     $$
     \begin{array}{l}
     (a_1 \vee \neg a_2) \\
     (a_1 \vee \neg a_3) \\
     (a_3 \vee \neg a_2) \\
     (a_3 \vee \neg a_1) \\
     (\neg a_1 \vee \neg a_3)
     \end{array}
     $$
    \end{minipage}
    & ~~~~~~ &
    \begin{minipage}{.4\textwidth}
      \begin{tabular}{ccl}
	\hline
	Trail & dl & Reason \\
	\hline
	$a_1$ & 1 & Decision \\
	$a_3$ & 1 & $(a_3 \vee \neg a_1)$ \\
	\hline
      \end{tabular}
      \bigskip \\
      $\{ \dec{a_1}{1}, \dec{a_3}{1} \}$
    \end{minipage}
  \end{tabular}

  \vfill
  \pause

  \begin{minipage}{\textwidth}
    \begin{prooftree}
    \AxiomC{$(\neg a_1 \vee \neg a_3)$}
    \AxiomC{$( a_3 \vee \neg a_1 )$}
    \BinaryInfC{$(\neg a_1)$}
    \end{prooftree}
  \end{minipage}

  \vfill
  \pause
  Conflict clause: $(\neg a_1)$ \pause \\
  Backtracking level: 0
\end{frame}

\begin{frame}
  \frametitle{Lecture 4 - Exercize 1}

  \scriptsize

  \begin{tabular}{ccc}
    \begin{minipage}{.4\textwidth}
     $$
     \begin{array}{l}
     (a_1 \vee \neg a_2) \\
     (a_1 \vee \neg a_3) \\
     (a_3 \vee \neg a_2) \\
     (a_3 \vee \neg a_1) \\
     (\neg a_1 \vee \neg a_3) \\
     (\neg a_1) \\
     (a_1 \vee a_2 \vee a_3)
     \end{array}
     $$
    \end{minipage}
    & ~~~~~~ &
    \begin{minipage}{.4\textwidth}
      \begin{tabular}{ccl}
	\hline
	Trail & dl & Reason \\
	\hline
	$\neg a_1$ & 0 & $(\neg a_1)$ \\
	$\neg a_2$ & 0 & $(a_1 \vee \neg a_2)$ \\
	$\neg a_3$ & 0 & $(a_1 \vee \neg a_3)$ \\
	\hline
      \end{tabular}
      \bigskip \\
      $\{ \neg \dec{a_1}{0}, \neg \dec{a_2}{0}, \neg \dec{a_3}{0} \}$
    \end{minipage}
  \end{tabular}

  \vfill
  \pause

  \begin{minipage}{\textwidth}
    \begin{prooftree}
    \AxiomC{$(a_1 \vee a_2 \vee a_3)$}
    \AxiomC{$(a_3 \vee \neg a_1)$}
    \BinaryInfC{$(a_1 \vee a_2)$}
    \AxiomC{$(a_1 \vee \neg a_2)$}
    \BinaryInfC{$(a_1)$}
    \AxiomC{$(\neg a_1)$}
    \BinaryInfC{$\bot$}
    \end{prooftree}
  \end{minipage}

  \vfill
  \pause
  Conflict clause: $\bot$ \pause \\
  Backtracking level: 0
\end{frame}
}
  \end{center}

\end{frame}

\begin{frame}
  \frametitle{Plan of (the rest of) the course}

  \begin{itemize}
    \item The eager approach: solving bit-vectors
    \vfill
    \item Modern SAT-solvers: conflict analysis, clause learning, and heuristics
    \vfill
    \item The Lazy approach: generalities
    \vfill
    \item A theory-solver for \Idl
    \vfill
    \item A theory-solver for \Uf
    \vfill
    \item A theory-solver for \Lra
    \vfill
    \item (see if there is time left)
  \end{itemize}

\end{frame}


\begin{frame}
  \frametitle{Outline}
  \tableofcontents
\end{frame}

\section{SMT-LIB and \smtsolvers}

\subsection{SMT-LIB}

\begin{frame}[fragile]
  \frametitle{SMT-LIB (v2) \url{http://www.smtlib.org}}

  The SMT-LIB initiative 
  \begin{itemize}
    \item defines a standard input language for \smtsolvers
    \item defines theories and logics in which \formulae can be written
    \item collects benchmarks 
  \end{itemize}

  \vfill
  \pause
  The SMT-LIB language allows to write \formulae in a lisp-like format. E.g.:
  \begin{verbatim}
    (< (+ x y) 0)
    (= (f x y) (g z))
  \end{verbatim} 
  stand for $x + y < 0$ and $f(x,y) = g(z)$ respectively

  \vfill
  \pause
  An SMT-LIB file looks more similar to a {\bf set of commands} for
  an \smtsolver, rather then a logic formula

\end{frame}

\begin{frame}[fragile]
  \frametitle{SMT-LIB Theories}

  \scriptsize

  An SMT-LIB theory consists of some {\bf sorts}, (e.g., \SInt) 
  and of some functions (e.g., $-,+$). 
  Predicates are also considered functions, with codomain in \SBoo
  (e.g., $<,\leq$). \pause For instance

  \vfill
  \begin{tabular}{cc}
  \begin{minipage}{.4\textwidth}
  \tiny
  \begin{verbatim}
     (theory Ints

      :sorts ((Int 0))
     
      :funs ((NUMERAL Int)
             (- Int Int)
             (- Int Int Int :left-assoc)
             (+ Int Int Int :left-assoc) 
             (* Int Int Int :left-assoc)
             (div Int Int Int :left-assoc)
             (mod Int Int Int)
             (abs Int Int)
             (<= Int Int Bool :chainable)
             (<  Int Int Bool :chainable)
             (>= Int Int Bool :chainable)
             (>  Int Int Bool :chainable)
            )

       [...]
     )
   \end{verbatim}
   \end{minipage}
   &
   \begin{minipage}{.4\textwidth}
     \tiny
      \begin{verbatim}
      (theory Core

       :sorts ((Bool 0))

       :funs ((true Bool)  
              (false Bool)
              (not Bool Bool)
              (=> Bool Bool Bool :right-assoc)
              (and Bool Bool Bool :left-assoc)
              (or Bool Bool Bool :left-assoc)
              (xor Bool Bool Bool :left-assoc)
              (par (A) (= A A Bool :chainable))
              (par (A) (distinct A A Bool :pairwise))
              (par (A) (ite Bool A A A))
             )

       [...]
      )
      \end{verbatim}
      \end{minipage} \\
    \multicolumn{2}{c}{\tiny These definitions can be found at \url{www.smtlib.org}}
  \end{tabular}
  \vfill
  \pause

  The sorts and the function symbols declared in a theory are always {\bf interpreted}.
  This means that a to specify a model for a formula $\varphi$, we just need to specify 
  the assignment of the variables to the concrete values in the sorts.

\end{frame}

\begin{frame}[fragile]
  \frametitle{SMT-LIB Logics}

  \scriptsize
  The difference between ``logic'' and ``theory'' might look very subtle. 
  An SMT-LIB logic includes a theory definition, plus it describes some
  restrictions on how \formulae can be built. \pause

  \vfill
  \hspace{-25pt}
  \begin{tabular}{ccc}
  \begin{minipage}{.4\textwidth}
    \tiny
    \begin{verbatim}
    (logic QF_LIA
     
     :theories (Ints)
     
     :language 
     "Closed quantifier-free formulas built 
     over an arbitrary expansion of the
     Ints signature with free constant symbols, 
     but whose terms of sort Int are all linear, 
     that is, have no occurrences of the function 
     symbols *, /, div, mod, and abs, except as 
     specified the :extensions attribute.
     "
      
     :extensions
     "Terms with _concrete_ coefficients are also 
     allowed, that is, terms of the form c, (* c x), 
     or (* x c)  where x is a free constant and c 
     is a term of the form n or (- n) for some numeral n.
     "
    )
    \end{verbatim}
  \end{minipage}
  & ~~~~ & \pause
  \begin{minipage}{.4\textwidth}
    \tiny
    \begin{verbatim}
    (logic QF_IDL

     :theories (Ints)

     :language
     "Closed quantifier-free formulas with 
     atoms of the form:
     - q
     - (op (- x y) n),
     - (op (- x y) (- n)), or
     - (op x y)
     where
     - q is a variable or free constant symbol of sort Bool,
     - op is <, <=, >, >=, =, or distinct,
     - x, y are free constant symbols of sort Int, 
     - n is a numeral. 
     "
    )
    \end{verbatim}
  \end{minipage}
  \end{tabular}

  \vfill
  \pause

  In the following we will not be so strict, and we will not make any distinction 
  between ``theories'' and ``logics'', calling both ``theories''. \pause 
  For instance when we will say that we reason modulo the theory \Lia we mean 
  that we are working with {\tt QF\_LIA} \formulae

\end{frame}

\begin{frame}[fragile]
  \frametitle{Writing an SMT-LIB file} 
  \scriptsize

  The logic can be specified with the command
  \begin{verbatim}
    (set-logic QF_LIA)
  \end{verbatim}
  \vfill
  \pause
  Variables are declared with
  \begin{verbatim}
    (declare-fun x ( ) Int)
  \end{verbatim}
  \vfill
  \pause
  A formula is specified with 
  \begin{verbatim}
    (assert (<= (+ x y) 0))
  \end{verbatim}
  \vfill
  \pause
  Asks the tool to compute satisfiability of assertions
  \begin{verbatim}
    (check-sat)
  \end{verbatim}
  \vfill
  \pause
  Asks the tool to return a model (in case of sat result)
  \begin{verbatim}
    (set-option :produce-models true)
    ...
    (get-value (x y))
  \end{verbatim}
  \vfill
  \pause
  Disable annoying printouts
  \begin{verbatim}
    (set-option :print-success false)
  \end{verbatim}

\end{frame}

\begin{frame}[fragile]
  \frametitle{Example}

  \begin{verbatim}
    (set-logic QF_LIA)
    (declare-fun x ( ) Int)
    (declare-fun y ( ) Int)
    (declare-fun a ( ) Bool)
    (assert (<= (+ x y) 0))
    (assert (= x 0))
    (assert (or (not a) (= x 1) (>= y 0)))
    (assert (not (= (+ y 1) 0)))
    (check-sat)
    (exit)
  \end{verbatim} 

  which stands for the \Lia formula
  $$
  (x + y \leq 0) \wedge (x = 0) \wedge ((\neg a \vee (x = 1) \vee (y \geq 0)) \wedge \neg(y + 1 = 0)
  $$

\end{frame}

\subsection{\smtsolvers}

\begin{frame}[fragile]
  \frametitle{\smtsolvers}

  An \smtsolver is a tool that can parse and solve an SMT-LIB
  benchmark.
  \vfill
  \pause
  There are many such tools available online. In this course
  we will use \yices (developed at SRI, Stanford, closed source), \zthree 
  (developed at MSR, Redmond, closed source) and \opensmt (developed here, open source). \pause
  Other available tools are \mathsat, \cvcfour, \boolector, \verit, \stp.
  \vfill
  \pause
  {\scriptsize
  \begin{verbatim}
    roberto@moriarty:examples$ smtlib2yices < test1.smt2 
    success
    success
    success
    success
    success
    success
    success
    success
    sat
  \end{verbatim}  
  }

\end{frame}

\begin{frame}[fragile]
  \frametitle{SMT-LIB script}

  The SMT-LIB language allows specification of {\bf scripts}. A script
  is a benchmark that may contain many {\tt check-sat} commands. Also, 
  it may include {\tt push} and {\tt pop} commands which
  can be used to control the assertion stack

  {\scriptsize
  \begin{verbatim}
    (set-option :print-success false)
    (set-logic QF_LIA)
    (declare-fun x ( ) Int)
    (declare-fun y ( ) Int)
    (assert (<= (+ x y) 0))
    (assert (= x 0))
    (assert (or (= x 1) (>= y 0)))
    (check-sat)
    (push 1)
    (assert (not (= y 0)))
    (check-sat)
    (pop 1)
    (check-sat)
    (exit)
  \end{verbatim} 
  }

\end{frame}

\begin{frame}
  \frametitle{Exercizes}

  \begin{enumerate}
    \item Translate the following \Lia formula SMT-LIB, and evaluate it with an \smtsolver
    $$(x - y \leq 0) \wedge (y - z \leq 0) \wedge ((z - x \leq -1) \vee (z - x \leq -2))$$
    \vfill

    \item Translate the following \Lra formula SMT-LIB, and evaluate it with an \smtsolver
    $$(b \vee (x + y \leq 0)) \wedge (\neg b \vee (x + z \leq 10))$$
    \vfill

    \item For the satisfiable \formulae above print out a model
    \vfill

    \item For the satisfiable \formulae above, add constraints such that they become unsatisfiable

  \end{enumerate}

\end{frame}


\end{document}
