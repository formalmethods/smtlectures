\subsection{The DPLL Procedure}

\begin{frame}
  \frametitle{The DPLL Procedure (Revised)}

  \scriptsize

  \begin{tabbing}
  as \= a \= a \= as \= asasdfsdfasdfasdfasdfasdfasdfasdfasdfasdf \= \kill
  DPLL( $V$, $\claset$ ) \\ \\
  \> if ( ``a clause has all $\bot$ literals'' ) return $unsat$; \> \> \> \>  // Failure \\
  \> if ( ``all clauses has a $\top$ literal'' ) return $sat$; \> \> \> \>  // Model found \\ \\
  \pause
  \> if ( ``a clause has all but one literal $l$ to $\bot$'' )  \\
  \> \> return DPLL( $V \cup \{ l \}$, $\claset$ );   \> \> \>               // BCP \\ \\
  \pause
  \> $l =$ ``pick an unassigned literal'' \> \> \> \>                          // Decision \\ \\
  \pause
  \> if ( DPLL( $V \cup \{ l \}$, $\claset$ ) == $sat$ )        \> \> \> \>  // Left Branch\\
  \> \> return $sat$; \\
  \> else \\
  \> \> return DPLL( $V \cup \{ \neg l \}$, $\claset$ );           \> \> \>  // Backtracking (Rigth Branch)
  \end{tabbing}

  \vfill
  \pause

  To be called as DPLL( $\{\ \}$, $\claset$ )

\end{frame}

\subsection{The Iterative DPLL Procedure}

\begin{frame}
  \frametitle{The Iterative DPLL Procedure (Revised)}

  \scriptsize

  \vfill

  \begin{tabbing}
  as \= a \= a \= a \= as \= asdfsdfasdfasdfasdfasdfasdfasdfasdfasdf \= \kill
  \> $dl = 0$; $flipped = \{\ \}$; $trail = \{\ \}$;    \> \> \> \> \> // Decision level, flipped vars, assignment\\ \\
  \> while ( $true$ ) \\ \\
  \> \> if ( BCP( ) == conflict )                          \> \> \> \> // Do BCP until possible \\
  \> \> \> $done = false$; \\
  \> \> \> do \\
  \> \> \> \> if ( $dl == 0$ ) return $unsat$;                   \> \> // Unresolvable conflict \\
  \> \> \> \> $l = $ {\sc GetDecisionLiteralAt}( $dl$ ); \\
  \> \> \> \> {\sc BacktrackTo}( $dl - 1$ );                         \> \> // Backtracking (shrinks $trail$) \\
  \> \> \> \> if ( var($l$) $\in flipped$ ) \\
  \> \> \> \> \> $dl = dl - 1$; \\
  \> \> \> \> else \\
  \> \> \> \> \> $trail = trail \cup \{ \neg l \}$; \\ 
  \> \> \> \> \> $flipped = flipped \cup \{ $var($l$)$ \}$; \\ 
  \> \> \> \> \> $done = true$; \\
  \> \> \> while ( !$done$ ); \\ \\
  \> \> else \\
  \> \> \> if ( ``all variables assigned'' ) return $sat$;    \> \> \> // $trail$ holds satisfying assignment \\
  \> \> \> $l = $ {\sc Decision}( );                                \> \> \> // Do another decision \\
  \> \> \> $trail = trail \cup \{ l \}$ \\
  \> \> \> $dl = dl + 1$;                                     \> \> \> // Increase decision level \\
  \end{tabbing}

  \vfill

\end{frame}

\begin{frame}
  \frametitle{The DPLL Procedure (Revised)}

  Some characteristics

  \begin{itemize}
    \vfill
    \item BCP is called with highest priority, as much as possible
    \vfill
    \item Backtracking is performed {\bf chronologically}: search backtracks to
          the previous decision-level
    \vfill
  \end{itemize}

  \pause
  Some differences w.r.t. ``original'' DPLL Procedure

  \begin{itemize}
    \vfill 
    \item $\claset$ is not touched: non-destructive data-structures (memory efficient)
    \vfill 
    \item Pure Literal Rule is not used: expensive to implement (!)
  \end{itemize}

\end{frame}
