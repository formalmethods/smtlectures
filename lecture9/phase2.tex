\begin{frame}[fragile]
  \frametitle{Phase 2 - Connecting the \tsolver}

  \begin{enumerate}[1.]

    \item Create a new logic

      \begin{enumerate}[a.]

	\vfill

	\item \verb|$ vim src/common/Global.h|

	\vfill

	\item Add \verb|, QF_SO| around line 196

	\vfill

	\item Add \verb|else if ( l == QF_SO ) return "QF_SO";| 
	      around line 312

	\vfill

	\item Add \verb|using opensmt::QF_SO;| 
	      around line 346

	\vfill

	\item \verb|$ vim src/api/OpenSMTContext.C| 

	\vfill
	  
	\item Add \verb|else if ( strcmp( str, "QF_SO" ) == 0 ) l = QF_SO;|
	      around line 88

	\vfill
	  
	\item Compile again \verb|$ cd debug; make; cd ..| (to see
	      if any typo was introduced)

      \end{enumerate}

  \end{enumerate} 

\end{frame}

\begin{frame}[fragile]
  \frametitle{Phase 2 - Connecting the \tsolver}

  \begin{enumerate}[2.]

    \item Allocate the solver

      \begin{enumerate}[a.]

	\scriptsize

	\vfill

	\item \verb|$ vim src/egraph/EgraphSolver.C|

	\vfill

	\item Add \verb|#include "SOSolver.h"| around line 29

	\vfill

	\item Add around line 853
	\begin{verbatim}
else if ( config.logic == QF_SO )
{
  tsolvers.push_back( new SOSolver( tsolvers.size( )
                                  , "Simple Order Solver"
                                  , config
                                  , *this
                                  , sort_store
                                  , explanation
                                  , deductions
                                  , suggestions ) );
  #ifdef STATISTICS
    tsolvers_stats.push_back( new TSolverStats( ) );
  #endif
}
	\end{verbatim}

      \end{enumerate}

      \vfill

      \item Compile again \verb|$ cd debug; make; cd ..| (to see
	    if any typo was introduced)

  \end{enumerate} 

\end{frame}
