\section{Lecture 4 - The Lazy Approach}

\begin{frame}
  \frametitle{The Lazy Approach}

  \scriptsize

  Notice that
  \begin{itemize}
    \item Assignments $\mu$ of $\varphi$ are many (potentially $\infty$),
          infeasible to check if any of them is a model {\bf systematically}
    \item Models $\babst{\mu}$ of $\babst{\varphi}$ are finite in number,
          and easy to enumerate with a SAT-solver
    \item A model $\babst{\mu}$ is nothing but a {\bf conjunction of \tatoms},
          can be checked efficiently with a \tsolver
  \end{itemize}
  \vfill
  These observations suggest us a methodology
  to tackle the SMT(\T) problem
  \begin{itemize}
    \item Enumerate a Boolean model $\babst{\mu}$ of $\babst{\varphi}$ (abstraction). If no model 
	  exist we are done ($\varphi$ is unsatisfiable) 
    \item Check if $\babst{\mu}$ is satisfiable using the \tsolver. If so $\babst{\mu}$ can be extended 
          to a model $\mu$ of $\varphi$, and so we are done ! ($\varphi$ is satisfiable) 
    \item It not, we tell the SAT-solver not to enumerate $\babst{\mu}$ again,
          thus {\bf cutting away systematically an infinite number} 
	  of assignments for $\varphi$ (refinement) 
    \item It can be blocked by adding a clause $\neg \babst{\mu}$. Go up
    \item It terminates because there are finite Boolean models
  \end{itemize}

\end{frame}

\begin{frame}
  \frametitle{The Lazy Approach}

  \scriptsize
  
  The lazy approach falls into the so-called {\bf abstraction-refinement} 
  paradigm
  \vfill
  \begin{center}
  \scalebox{.5}{\subsection{Lazy Approach as Abstraction Refinement}

\begin{frame}
  \frametitle{Abstraction}
  Assigment relations
  \vfill
  \begin{center}
  \begin{tabular}{ccc}

    \begin{minipage}{.2\textwidth}

    $\babst{\varphi}$ \\
    \\
    \\
    \\
    \\
    \\
    $\varphi$ 

    \end{minipage}

    &

    \begin{minipage}{.55\textwidth}
      \begin{overlayarea}{\textwidth}{5cm}
	\only<1-3|handout:0>{\scalebox{.4}{\input{assignments_1.pdf_t}}}
	\only<4|handout:0>{\scalebox{.4}{\input{assignments_2.pdf_t}}}
	\only<5|handout:0>{\scalebox{.4}{\input{assignments_3.pdf_t}}}
	\only<6|handout:0>{\scalebox{.4}{\input{assignments_4.pdf_t}}}
	\only<7>{\scalebox{.4}{\input{assignments_5.pdf_t}}}
      \end{overlayarea}
    \end{minipage}

    &

    \begin{minipage}{.3\textwidth}

    \onslide<3->{$2^n$} \\
    \onslide<3->{($n = |\{ a_i \}|$)}\\
    \\
    \\
    \\
    \onslide<2->{$\infty$} \\ 
    \onslide<2->(can be) 

    \end{minipage}

  \end{tabular}
  \end{center}

\end{frame}

\begin{frame}
  \frametitle{Abstraction}
  Model relations
  \begin{itemize}
    \item<2-> if $\mu$ is a model for $\varphi$, then $\babst{\mu}$ is a model for $\babst{\varphi}$
    \item<3-> if $\babst{\mu}$ is not a model for $\babst{\varphi}$, then there is no $\mu$ that is a model for $\varphi$
    \item<4-> there may be some model $\babst{\mu}$ for $\babst{\varphi}$ that does not map to any model 
	      $\mu$ for $\varphi$
  \end{itemize}
  \vfill
  \begin{center}
  \begin{tabular}{cc}

    \begin{minipage}{.2\textwidth}

    $\babst{\varphi}$ \\
    \\
    \\
    \\
    \\
    \\
    $\varphi$ 

    \end{minipage}

    &

    \begin{minipage}{.7\textwidth}
      \begin{overlayarea}{\textwidth}{5cm}
	\only<1|handout:0>{\scalebox{.4}{\input{models_1.pdf_t}}}
	\only<2|handout:0>{\scalebox{.4}{\input{models_2.pdf_t}}}
	\only<3|handout:0>{\scalebox{.4}{\input{models_3.pdf_t}}}
	\only<4>{\scalebox{.4}{\input{models_4.pdf_t}}}
      \end{overlayarea}
    \end{minipage}

  \end{tabular}
  \end{center}

\end{frame}

\begin{frame}
  \frametitle{Abstraction Refinement}

  \scriptsize

  Notice that
  \begin{itemize}
    \item Assignments $\mu$ of $\varphi$ are many (potentially $\infty$),
          infeasible to check if any of them is a model {\bf systematically}
    \item Models $\babst{\mu}$ of $\babst{\varphi}$ are finite in number,
          and easy to enumerate with a SAT-solver
    \item A model $\babst{\mu}$ is nothing but a {\bf conjunction of \tatoms},
          can be checked efficiently with a \tsolver
  \end{itemize}
  \vfill
  \pause
  These observations suggest us a methodology
  to tackle the SMT(\T) problem
  \begin{itemize}
    \item Enumerate a Boolean model $\babst{\mu}$ of $\babst{\varphi}$ (abstraction). If no model 
	  exist we are done ($\varphi$ is unsatisfiable) \pause
    \item Check if $\babst{\mu}$ is satisfiable using the \tsolver. If so $\babst{\mu}$ can be extended 
          to a model $\mu$ of $\varphi$, and so we are done ! ($\varphi$ is satisfiable) \pause
    \item It not, we tell the SAT-solver not to enumerate $\babst{\mu}$ again,
          thus {\bf cutting away systematically an infinite number} 
	  of assignments for $\varphi$ (refinement) \pause
    \item It can be blocked by adding a clause $\neg \babst{\mu}$. Go up \pause
    \item It terminates because there are finite Boolean models
  \end{itemize}

\end{frame}

\begin{frame}
  \frametitle{Abstraction Refinement}

  \scriptsize
  
  The lazy approach falls into the so-called {\bf abstraction-refinement} 
  paradigm
  \vfill
  \begin{center}
  \scalebox{.5}{\subsection{Lazy Approach as Abstraction Refinement}

\begin{frame}
  \frametitle{Abstraction}
  Assigment relations
  \vfill
  \begin{center}
  \begin{tabular}{ccc}

    \begin{minipage}{.2\textwidth}

    $\babst{\varphi}$ \\
    \\
    \\
    \\
    \\
    \\
    $\varphi$ 

    \end{minipage}

    &

    \begin{minipage}{.55\textwidth}
      \begin{overlayarea}{\textwidth}{5cm}
	\only<1-3|handout:0>{\scalebox{.4}{\input{assignments_1.pdf_t}}}
	\only<4|handout:0>{\scalebox{.4}{\input{assignments_2.pdf_t}}}
	\only<5|handout:0>{\scalebox{.4}{\input{assignments_3.pdf_t}}}
	\only<6|handout:0>{\scalebox{.4}{\input{assignments_4.pdf_t}}}
	\only<7>{\scalebox{.4}{\input{assignments_5.pdf_t}}}
      \end{overlayarea}
    \end{minipage}

    &

    \begin{minipage}{.3\textwidth}

    \onslide<3->{$2^n$} \\
    \onslide<3->{($n = |\{ a_i \}|$)}\\
    \\
    \\
    \\
    \onslide<2->{$\infty$} \\ 
    \onslide<2->(can be) 

    \end{minipage}

  \end{tabular}
  \end{center}

\end{frame}

\begin{frame}
  \frametitle{Abstraction}
  Model relations
  \begin{itemize}
    \item<2-> if $\mu$ is a model for $\varphi$, then $\babst{\mu}$ is a model for $\babst{\varphi}$
    \item<3-> if $\babst{\mu}$ is not a model for $\babst{\varphi}$, then there is no $\mu$ that is a model for $\varphi$
    \item<4-> there may be some model $\babst{\mu}$ for $\babst{\varphi}$ that does not map to any model 
	      $\mu$ for $\varphi$
  \end{itemize}
  \vfill
  \begin{center}
  \begin{tabular}{cc}

    \begin{minipage}{.2\textwidth}

    $\babst{\varphi}$ \\
    \\
    \\
    \\
    \\
    \\
    $\varphi$ 

    \end{minipage}

    &

    \begin{minipage}{.7\textwidth}
      \begin{overlayarea}{\textwidth}{5cm}
	\only<1|handout:0>{\scalebox{.4}{\input{models_1.pdf_t}}}
	\only<2|handout:0>{\scalebox{.4}{\input{models_2.pdf_t}}}
	\only<3|handout:0>{\scalebox{.4}{\input{models_3.pdf_t}}}
	\only<4>{\scalebox{.4}{\input{models_4.pdf_t}}}
      \end{overlayarea}
    \end{minipage}

  \end{tabular}
  \end{center}

\end{frame}

\begin{frame}
  \frametitle{Abstraction Refinement}

  \scriptsize

  Notice that
  \begin{itemize}
    \item Assignments $\mu$ of $\varphi$ are many (potentially $\infty$),
          infeasible to check if any of them is a model {\bf systematically}
    \item Models $\babst{\mu}$ of $\babst{\varphi}$ are finite in number,
          and easy to enumerate with a SAT-solver
    \item A model $\babst{\mu}$ is nothing but a {\bf conjunction of \tatoms},
          can be checked efficiently with a \tsolver
  \end{itemize}
  \vfill
  \pause
  These observations suggest us a methodology
  to tackle the SMT(\T) problem
  \begin{itemize}
    \item Enumerate a Boolean model $\babst{\mu}$ of $\babst{\varphi}$ (abstraction). If no model 
	  exist we are done ($\varphi$ is unsatisfiable) \pause
    \item Check if $\babst{\mu}$ is satisfiable using the \tsolver. If so $\babst{\mu}$ can be extended 
          to a model $\mu$ of $\varphi$, and so we are done ! ($\varphi$ is satisfiable) \pause
    \item It not, we tell the SAT-solver not to enumerate $\babst{\mu}$ again,
          thus {\bf cutting away systematically an infinite number} 
	  of assignments for $\varphi$ (refinement) \pause
    \item It can be blocked by adding a clause $\neg \babst{\mu}$. Go up \pause
    \item It terminates because there are finite Boolean models
  \end{itemize}

\end{frame}

\begin{frame}
  \frametitle{Abstraction Refinement}

  \scriptsize
  
  The lazy approach falls into the so-called {\bf abstraction-refinement} 
  paradigm
  \vfill
  \begin{center}
  \scalebox{.5}{\subsection{Lazy Approach as Abstraction Refinement}

\begin{frame}
  \frametitle{Abstraction}
  Assigment relations
  \vfill
  \begin{center}
  \begin{tabular}{ccc}

    \begin{minipage}{.2\textwidth}

    $\babst{\varphi}$ \\
    \\
    \\
    \\
    \\
    \\
    $\varphi$ 

    \end{minipage}

    &

    \begin{minipage}{.55\textwidth}
      \begin{overlayarea}{\textwidth}{5cm}
	\only<1-3|handout:0>{\scalebox{.4}{\input{assignments_1.pdf_t}}}
	\only<4|handout:0>{\scalebox{.4}{\input{assignments_2.pdf_t}}}
	\only<5|handout:0>{\scalebox{.4}{\input{assignments_3.pdf_t}}}
	\only<6|handout:0>{\scalebox{.4}{\input{assignments_4.pdf_t}}}
	\only<7>{\scalebox{.4}{\input{assignments_5.pdf_t}}}
      \end{overlayarea}
    \end{minipage}

    &

    \begin{minipage}{.3\textwidth}

    \onslide<3->{$2^n$} \\
    \onslide<3->{($n = |\{ a_i \}|$)}\\
    \\
    \\
    \\
    \onslide<2->{$\infty$} \\ 
    \onslide<2->(can be) 

    \end{minipage}

  \end{tabular}
  \end{center}

\end{frame}

\begin{frame}
  \frametitle{Abstraction}
  Model relations
  \begin{itemize}
    \item<2-> if $\mu$ is a model for $\varphi$, then $\babst{\mu}$ is a model for $\babst{\varphi}$
    \item<3-> if $\babst{\mu}$ is not a model for $\babst{\varphi}$, then there is no $\mu$ that is a model for $\varphi$
    \item<4-> there may be some model $\babst{\mu}$ for $\babst{\varphi}$ that does not map to any model 
	      $\mu$ for $\varphi$
  \end{itemize}
  \vfill
  \begin{center}
  \begin{tabular}{cc}

    \begin{minipage}{.2\textwidth}

    $\babst{\varphi}$ \\
    \\
    \\
    \\
    \\
    \\
    $\varphi$ 

    \end{minipage}

    &

    \begin{minipage}{.7\textwidth}
      \begin{overlayarea}{\textwidth}{5cm}
	\only<1|handout:0>{\scalebox{.4}{\input{models_1.pdf_t}}}
	\only<2|handout:0>{\scalebox{.4}{\input{models_2.pdf_t}}}
	\only<3|handout:0>{\scalebox{.4}{\input{models_3.pdf_t}}}
	\only<4>{\scalebox{.4}{\input{models_4.pdf_t}}}
      \end{overlayarea}
    \end{minipage}

  \end{tabular}
  \end{center}

\end{frame}

\begin{frame}
  \frametitle{Abstraction Refinement}

  \scriptsize

  Notice that
  \begin{itemize}
    \item Assignments $\mu$ of $\varphi$ are many (potentially $\infty$),
          infeasible to check if any of them is a model {\bf systematically}
    \item Models $\babst{\mu}$ of $\babst{\varphi}$ are finite in number,
          and easy to enumerate with a SAT-solver
    \item A model $\babst{\mu}$ is nothing but a {\bf conjunction of \tatoms},
          can be checked efficiently with a \tsolver
  \end{itemize}
  \vfill
  \pause
  These observations suggest us a methodology
  to tackle the SMT(\T) problem
  \begin{itemize}
    \item Enumerate a Boolean model $\babst{\mu}$ of $\babst{\varphi}$ (abstraction). If no model 
	  exist we are done ($\varphi$ is unsatisfiable) \pause
    \item Check if $\babst{\mu}$ is satisfiable using the \tsolver. If so $\babst{\mu}$ can be extended 
          to a model $\mu$ of $\varphi$, and so we are done ! ($\varphi$ is satisfiable) \pause
    \item It not, we tell the SAT-solver not to enumerate $\babst{\mu}$ again,
          thus {\bf cutting away systematically an infinite number} 
	  of assignments for $\varphi$ (refinement) \pause
    \item It can be blocked by adding a clause $\neg \babst{\mu}$. Go up \pause
    \item It terminates because there are finite Boolean models
  \end{itemize}

\end{frame}

\begin{frame}
  \frametitle{Abstraction Refinement}

  \scriptsize
  
  The lazy approach falls into the so-called {\bf abstraction-refinement} 
  paradigm
  \vfill
  \begin{center}
  \scalebox{.5}{\input{ar.pdf_t}}
  \end{center}

\end{frame}
}
  \end{center}

\end{frame}
}
  \end{center}

\end{frame}
}
  \end{center}

\end{frame}

\begin{frame}
  \frametitle{The Lazy Approach}

  \scriptsize

  The interaction described naturally falls within the
  CDCL style, enriched with a \tsolver
      $$\varphi \equiv 
      (x=3 \vee \neg (x<3))\ \swedge\
      (x=3 \vee \neg (x>3))\ \swedge\
      (x>3 \vee \neg (x<3))\ \swedge\
      (x>3 \vee \neg (x=3))$$ 
  \vfill
  \begin{columns}

    \begin{column}{.5cm}
      \vspace{-165pt}
      $\babst{\varphi} \equiv$
    \end{column}

    \begin{column}{5cm}
      $(\coltwoat{\coloneat{a_1}{8-13}}{2-6} \vee \coltwoat{\neg a_2}{9-13})$ \\
      $(\coltwoat{\coloneat{a_1}{8-13}}{2-6} \vee \coltwoat{\coloneat{\neg a_3}{3-6}}{10-13})$ \\
      $(\coloneat{\coltwoat{a_3}{3-6}}{10-13} \vee \coltwoat{\neg a_2}{9-13})$ \\
      $(\coloneat{\coltwoat{a_3}{3-6}}{10-13} \vee \coloneat{\coltwoat{\neg a_1}{8-13}}{2-6})$ \\
      \onslide<6->{$(\coloneat{\coltwoat{\neg a_1}{10-13}}{6} \vee \coltwoat{\coloneat{\neg a_3}{6}}{10-13})$ \\}
      \onslide<7->{$(\coltwoat{\neg a_1}{8-13})$ \\}
      \onslide<13->{$(\coloneat{a_1}{13} \vee \coloneat{a_2}{13} \vee \coloneat{a_3}{13})$ \\}
      \onslide<14->{$(\ )$ \\}
      \bigskip
      $a_1 \equiv x=3$ \\
      $a_2 \equiv x<3$ \\
      $a_3 \equiv x>3$ \\
      \bigskip
      $\babst{\mu}$: $\{\ \only<2-6|handout:0>{a_1}\only<3-6|handout:0>{, a_3}\only<8-13|handout:0>{\neg a_1}\only<9-13|handout:0>{, \neg a_2}\only<10-13|handout:0>{, \neg a_3}\ \}$ \\
      \bigskip
      \begin{tabular}{rl}
      SAT-solver: & \only<1,4-5,11-12|handout:0>{Idle}\only<2|handout:0>{Decision}\only<3,8-10|handout:0>{BCP}\only<6,13|handout:0>{Learn}\only<7,14|handout:0>{Conf. Analysis, Backtrack}\only<15>{UNS} \\
        \tsolver: & \only<1,2,3,6-10,13->{Idle}\only<4,5,11,12|handout:0>{Is $\babst{\mu}$ \T-satisfiable ?}\only<5,12|handout:0>{ NO}
      \end{tabular}
    \end{column}

    \begin{column}{5cm}
      \begin{overlayarea}{5cm}{5cm}
	\only<1|handout:0>{\scalebox{.6}{\input{search_0.pdf_t}}}
	\only<2|handout:0>{\scalebox{.6}{\input{search_1.pdf_t}}}
	\only<3,4,5|handout:0>{\scalebox{.6}{\input{search_2.pdf_t}}}
	\only<6,7|handout:0>{\scalebox{.6}{\input{search_3.pdf_t}}}
	\only<8|handout:0>{\scalebox{.6}{\input{search_4.pdf_t}}}
	\only<9|handout:0>{\scalebox{.6}{\input{search_5.pdf_t}}}
	\only<10-12|handout:0>{\scalebox{.6}{\input{search_6.pdf_t}}}
	\only<13->{\scalebox{.6}{\input{search_7.pdf_t}}}
      \end{overlayarea}
    \end{column}

  \end{columns}

\end{frame}

\begin{frame}
  \frametitle{Lecture 4 - Exercize 1}

  \scriptsize

  \begin{tabular}{ccc}
    \begin{minipage}{.4\textwidth}
     $$
     \begin{array}{l}
     (a_1 \vee \neg a_2) \\
     (a_1 \vee \neg a_3) \\
     (a_3 \vee \neg a_2) \\
     (a_3 \vee \neg a_1) \\
     (\neg a_1 \vee \neg a_3)
     \end{array}
     $$
    \end{minipage}
    & ~~~~~~ &
    \begin{minipage}{.4\textwidth}
      \begin{tabular}{ccl}
	\hline
	Trail & dl & Reason \\
	\hline
	$a_1$ & 1 & Decision \\
	$a_3$ & 1 & $(a_3 \vee \neg a_1)$ \\
	\hline
      \end{tabular}
      \bigskip \\
      $\{ \dec{a_1}{1}, \dec{a_3}{1} \}$
    \end{minipage}
  \end{tabular}

  \vfill
  \pause

  \begin{minipage}{\textwidth}
    \begin{prooftree}
    \AxiomC{$(\neg a_1 \vee \neg a_3)$}
    \AxiomC{$( a_3 \vee \neg a_1 )$}
    \BinaryInfC{$(\neg a_1)$}
    \end{prooftree}
  \end{minipage}

  \vfill
  \pause
  Conflict clause: $(\neg a_1)$ \pause \\
  Backtracking level: 0
\end{frame}

\begin{frame}
  \frametitle{Lecture 4 - Exercize 1}

  \scriptsize

  \begin{tabular}{ccc}
    \begin{minipage}{.4\textwidth}
     $$
     \begin{array}{l}
     (a_1 \vee \neg a_2) \\
     (a_1 \vee \neg a_3) \\
     (a_3 \vee \neg a_2) \\
     (a_3 \vee \neg a_1) \\
     (\neg a_1 \vee \neg a_3) \\
     (\neg a_1) \\
     (a_1 \vee a_2 \vee a_3)
     \end{array}
     $$
    \end{minipage}
    & ~~~~~~ &
    \begin{minipage}{.4\textwidth}
      \begin{tabular}{ccl}
	\hline
	Trail & dl & Reason \\
	\hline
	$\neg a_1$ & 0 & $(\neg a_1)$ \\
	$\neg a_2$ & 0 & $(a_1 \vee \neg a_2)$ \\
	$\neg a_3$ & 0 & $(a_1 \vee \neg a_3)$ \\
	\hline
      \end{tabular}
      \bigskip \\
      $\{ \neg \dec{a_1}{0}, \neg \dec{a_2}{0}, \neg \dec{a_3}{0} \}$
    \end{minipage}
  \end{tabular}

  \vfill
  \pause

  \begin{minipage}{\textwidth}
    \begin{prooftree}
    \AxiomC{$(a_1 \vee a_2 \vee a_3)$}
    \AxiomC{$(a_3 \vee \neg a_1)$}
    \BinaryInfC{$(a_1 \vee a_2)$}
    \AxiomC{$(a_1 \vee \neg a_2)$}
    \BinaryInfC{$(a_1)$}
    \AxiomC{$(\neg a_1)$}
    \BinaryInfC{$\bot$}
    \end{prooftree}
  \end{minipage}

  \vfill
  \pause
  Conflict clause: $\bot$ \pause \\
  Backtracking level: 0
\end{frame}
