\section{Lecture 7 - A \tsolver for \Uf}

\begin{frame}
  \frametitle{A \tsolver for \Uf}

  The solving phase inside the \Uf-solver happens in two steps
  \vfill
  Given a conjunction of \tlits $\varphi$ to check for satisfiability, the \Uf-solver
  \vfill
  first it constructs {\bf equivalence classes} using $\varphi^+$ (the set of positive \tlits),
  using, for example, the Union-Find algorithm
  \vfill
  and then checks one by one the negative \tlits in $\varphi^-$

\end{frame}

\begin{frame}
  \frametitle{A \tsolver for \Uf}

  \scriptsize

  First of all notice that \tconflicts are always of the form
  $$
    \{\ s \not= t, \mbox{``other equalities that cause } s \mbox{ and } t \mbox{ to join the same class''} \} 
  $$
  We reconstruct the conflict by storing the reason that caused two classes to collapse. When a
  $s \not=t $ causes $unsat$, we call a routine Explain($s$,$t$) that collects all the reasons
  that made $s$ equal to $t$ \\
  Example 
  $$
  \{\ \coloneat{x\!=\!y}{2}
   ,\ \coloneat{y\!=\!w}{3}
   ,\ \coloneat{f(x)\!=\!z}{5}
   ,\ \coloneat{f(w)\!=\!a}{6}
   ,\ \coloneat{a\!\not=\!z}{7-}\ \}
  $$
  \begin{overlayarea}{\textwidth}{3cm}
    \begin{center}
    \only<1|handout:0>{\input{confl}}
    \only<2|handout:0>{
\begin{tikzpicture}[auto]

%
% Styles
%
\tikzstyle{vertex} = [rectangle,draw=oproverblue!60,fill=oproverblue!20,thick]

\node[vertex] (y) at  (  2, 0)  {$y$};
\node[vertex] (x) at  (  4, 0)  {$x$};
\node[vertex] (w) at  (  6, 0)  {$w$};
\node[vertex] (fx) at (  3, 2)  {$f(x)$};
\node[vertex] (fw) at (  5, 2)  {$f(w)$};
\node[vertex] (z) at  (  1, 2)  {$z$};
\node[vertex] (a) at  (  7, 2)  {$a$};

\draw[->] (y)  -- node {$x\!=\!y$} (x);
\path (x)  edge [loop above] (x);
\path (z)  edge [loop above] (z);
\path (w)  edge [loop above] (w);
\path (a)  edge [loop above] (a);
\path (fx) edge [loop above] (fx);
\path (fw) edge [loop above] (fw);

\end{tikzpicture}
}
    \only<3|handout:0>{\input{confl_2}}
    \only<4|handout:0>{\input{confl_3}}
    \only<5|handout:0>{\input{confl_4}}
    \only<6->{\input{confl_5}}
    \end{center}
  \end{overlayarea}
  \vfill
  \onslide<7->{
  On processing $a\not=z$, we call Explain( $a$, $z$ )
  }

\end{frame}

\begin{frame}[fragile]
  \frametitle{Lecture 7 - Exercize 2}
  
  \scriptsize

  We can store a reason of a Union inside the struct \verb|Node|

  \begin{verbatim}
  struct Node { 
    ...
    Node *  root;     // Points to class' representant
    Enode * reason;   // T-atom that caused union
  };
  \end{verbatim} 

  \begin{tabbing}
  aa \= a \= a \= a \= xxxxxxxxxxxxxxaaaaaaaaaaaaaaaaaaaaa \= asdasdasdasdasdasdasdasdasd \kill

     \> explanation $= \{\ \}$ \> \> \> \> // Global variable \\
   \\
   1 \> {\bf procedure} Explain( $s$, $t$ ) \\
   \\
   2 \> if ( $s == t$ ) return           \> \> \> \> // Nothing to explain \ldots \\
   \\
   3 \> while( $s.next \not= s$ )        \> \> \> \> // Collect reasons while moving to root \\
   4 \> \> if ( $s.reason == $ NULL )       \> \> \> // No reason: union caused by congruence \\
     \> \> \> // Let $s \equiv f( s_1, \ldots, s_n )$, $s.next \equiv f( t_1, \ldots, t_n )$ \\
   5 \> \> \> foreach $i \in \{ 1, \ldots, n \}$ \\
   6 \> \> \> \> Explain( $s_i$, $t_i$ ) \\
   7 \> \> else                             \> \> \> // Reason exists, save it \\
   8 \> \> \> explanation $=$ explanation $\cup\ \{ s.reason \}$ \\ 
   9 \> \> $s = s.next$ \\
   \\
   // Same ``while'' loop for $t$ \ldots

  \end{tabbing}

\end{frame}
